% Created 2015-01-24 Sat 23:25
\documentclass[fleqn]{article}
\usepackage[utf8]{inputenc}
\usepackage[T1]{fontenc}
\usepackage{fixltx2e}
\usepackage{graphicx}
\usepackage{longtable}
\usepackage{float}
\usepackage{wrapfig}
\usepackage{rotating}
\usepackage[normalem]{ulem}
\usepackage{amsmath}
\usepackage{textcomp}
\usepackage{marvosym}
\usepackage{wasysym}
\usepackage{amssymb}
\usepackage{hyperref}
\tolerance=1000
\usepackage[utf8]{inputenc}
\usepackage[usenames,dvipsnames]{color}
\usepackage{a4wide}
\usepackage[backend=bibtex, style=numeric]{biblatex}
\usepackage{commath}
\usepackage{tikz}
\usepackage{amsmath}
\usetikzlibrary{shapes,backgrounds}
\usepackage{marginnote}
\usepackage{enumerate}
\usepackage{listings}
\usepackage{color}
\hypersetup{urlcolor=blue}
\hypersetup{colorlinks,urlcolor=blue}
\addbibresource{bibliography.bib}
\setlength{\parskip}{16pt plus 2pt minus 2pt}
\definecolor{codebg}{rgb}{0.96,0.99,0.8}
\DeclareMathOperator{\Sp}{Sp}
\DeclareMathOperator{\Solutions}{P}
\DeclareMathOperator{\Dim}{dim}
\DeclareMathOperator{\Image}{Im}
\DeclareMathOperator{\Ker}{Ker}
\author{Oleg Sivokon}
\date{\textit{<2015-01-24 Sat>}}
\title{Assignment 15, Linear Algebra 1}
\hypersetup{
  pdfkeywords={Assignment, Linear Algebra},
  pdfsubject={Fourth asssignment in the course Linear Algebra 1},
  pdfcreator={Emacs 25.0.50.1 (Org mode 8.2.2)}}
\begin{document}

\maketitle
\tableofcontents


\lstset{ %
  backgroundcolor=\color{codebg},
  basicstyle=\ttfamily\scriptsize,
  breakatwhitespace=false,         % sets if automatic breaks should only happen at whitespace
  breaklines=false,
  captionpos=b,                    % sets the caption-position to bottom
  commentstyle=\color{mygreen},    % comment style
  framexleftmargin=10pt,
  xleftmargin=10pt,
  framerule=0pt,
  frame=tb,                        % adds a frame around the code
  keepspaces=true,                 % keeps spaces in text, useful for keeping indentation of code (possibly needs columns=flexible)
  keywordstyle=\color{blue},       % keyword style
  showspaces=false,                % show spaces everywhere adding particular underscores; it overrides 'showstringspaces'
  showstringspaces=false,          % underline spaces within strings only
  showtabs=false,                  % show tabs within strings adding particular underscores
  stringstyle=\color{codestr},     % string literal style
  tabsize=2,                       % sets default tabsize to 2 spaces
}

 \clearpage

\section{Problems}
\label{sec-1}

\subsection{Problem 1}
\label{sec-1-1}

Check whether the following transformations are linear:

\begin{enumerate}
\item $T_1 : \mathbb{R}^2 \to \mathbb{R}^2, T_1(x, y)=(\sin{x}, y)$.
\item $T_2 : \mathbb{R} \to \mathbb{R}, T_2(x)=\abs{x}$.
\item $T_3 : \mathbb{R}[x] \to \mathbb{R}[x], T_3(p(x))=(x+1)p'(x)-p(x)$.
\end{enumerate}

\subsubsection{Answers 1, 2, 3}
\label{sec-1-1-1}
\begin{enumerate}
\item $T_1$ is not a linear transformation because it is not a bijection: it will
send any multiple of $\pi$ to the same value in the image.
\item $T_2$ is not a linear transformation by the same reasoning: it fails to be
a bijection because it will send every value in the domain and its multiple
with $-1$ to the same value in the image.
\item $T_3$, if I understood it correctly (there was a typo in the assignment
description---a missing parenthesis), is indeed a linear transformation:
the derivative (I believe that $p'$ is a first order derivative, but it 
could be anything else really as long as it's some kind of polynomial of
a finite degree) will be dominated by the polynomial of which it is a 
derivative, else it will be some finite order polynomial, in which case it
is still possible to find a bijectin and a total function.
\end{enumerate}
\subsection{Problem 2}
\label{sec-1-2}
Does there exist a linear non-zero transformation $T:\mathbb{R}^3\to\mathbb{R}^3$
such that $T(1,0,1)=T(1,2,1)=T(0,1,1)=T(2,3,3)$?  Prove an example if exists,
else explain in detail why it doesn't.

\subsubsection{Answer 4}
\label{sec-1-2-1}
No, such transformation doesn't exist.  The easiest way to see why is by
attempting to construct one.  In order to do so, let's give names to the
elements of the first row of the matrix representing the transformation $T$.
\emph{(We will only need the first row because the same equation will be obtained}
\emph{by multiplying any row of the matrix with a vector)}.  Hence, to reduce
the verbosity of the proof, let $a,b,c$ be the elements of the first row of
$T$.  Then $T(1,0,1)_1=a+c$ \emph{(I will use subscripts to denote the elements}
\emph{within vectors)}.  Other elements of the vector $T(1,0,1)$ will be, similarly,
the sums of the first and the last elements of the consequent rows of $T$.

Similarly, we find that $T(1,2,1)_1=a+2b+c$, $T(0,1,1)_1=b+c$,
$T(2,3,3)_1=2a+3b+3c$.  Now, since we know that the vectors are equal (from
the given), we may equate them.  Thus obtains:

\begin{align*}
  a+c      &= a+2b+c \\
  a+c      &= b+c \\
  2a+3b+3c &= a+2b+c
\end{align*}

solving these two equations gives us that $a=b$ and $2b=0$, hence $b=0$,
then, $a+b+2c=0$, which then gives $2c=0$, hence $c=0$.  In other words
the only transformation which satisfies the given condition is the
transformation, where all elements of the matrix representing it are zeros.
\subsection{Problem 3}
\label{sec-1-3}
Given $\{v_1, v_2, \ldots, v_k\}$ is a linearly independent group in linear
vector space $V$.  Provided $T:V\to V$ prove or disprove:

\begin{enumerate}
\item If $\{T(v_1), T(v_2), \ldots, T(v_3)\}$ is linearly independent, then
      $\Dim(\Image(T))=k$.
\item If $\{T(v_1), T(v_2), \ldots, T(v_3)\}$ spans $V$, then $\dim{V}=k$.
\end{enumerate}

\subsubsection{Answer 5}
\label{sec-1-3-1}
This is not necessarily true.  There could be some other vector, let's call
it $v_{k+1} \in V$, linearly independent from
$\{T(v_1), T(v_2), \ldots, T(v_3)\}$ and yet $T(v_{k+1}) \in V$.  Let, for the
sake of simplicity, choose $k=2$ and $V=\Sp\{e_1,e_2,e_3\}$, where $e_i$ are
the vectors from the standard basis, $T$ would be the identity transformation.

It is easy to see that $T$ will send all $\{e_1,e_2,e_3\}$ back to themselves,
thus $\Dim(\Image(T))=3$, yet was assumed to be 2.  This completes the proof.
\subsubsection{Answer 6}
\label{sec-1-3-2}
First, observe that $k$ cannot be smaller than $\Dim{V}$, because the span
generated by $T$ will be at least the basis of $V$, and basis needs to have the
same dimension as the space of which it is a basis.  The domain of $T$ is also
given as linearly independant.  In other words, the dimension of its span
cannot be larger than that of $V$.  Since $T$ is a function, it cannot assign
more than one element in its image to an element in its domain.  Thus, the only
possible way this transformation could have the given properties is if it was
an injection, and this would mean that its domain can have at most $k$ linearly
independent vectors in it, hence $k=\dim{V}$.
\subsection{Problem 4}
\label{sec-1-4}
Given
\begin{equation*}
  A =
  \begin{bmatrix}
     1 & -2 & -2 \\
     2 & -1 &  5 \\
    -2 &  3 &  1 
  \end{bmatrix}
\end{equation*}

And linear transformation $T:\textbf{M}^{\mathbb{R}}_{3 \times 3}
   \to\textbf{M}^{\mathbb{R}}_{3 \times 3}$ defined as $T(X)=AX$ for all
$X \in \textbf{M}^{\mathbb{R}}_{3 \times 3}$.
Let $T_A:\mathbb{R}^3\to\mathbb{R}^3$ be the matrix defined using $A$, in other
words $\forall x \in \mathbb{R}^3 : T_A(x)=Ax$.

\begin{enumerate}
\item Find basis for $\Ker{T_A}$ and $\Image{T_A}$.
\item Prove that $T$ is not invertible.
\item Find basis for $\Ker{T}$ and $\Image{T}$.
\item Prove that if $Y \in \Image{T}$, then $\rho(Y) \leq \Dim(\Image(T_A))$.
\item Prove that if $Y \in \Ker{T}$, then $\rho(Y) \leq \Dim(\Ker(T_A))$.
\end{enumerate}
\subsection{Problem 5}
\label{sec-1-5}
Let $T:\mathbb{R}^4\to\mathbb{R}^4$ be a linear transformation such that
$\Dim(\Ker(T))>\Dim(\Image(T))$ and the matrix representing the transformation
$T$ with the basis $B=((1,1,1,1),(1,1,1,0),(1,1,0,0),(1,0,0,0))$ is given
by:
\begin{equation*}
  [T]_B =
  \begin{bmatrix}
    1 & 2 & 3 & 4 \\
    1 & a_1 & b_1 & c_1 \\
    1 & a_2 & b_2 & c_2 \\
    1 & a_3 & b_3 & c_3
  \end{bmatrix}
\end{equation*}

\begin{enumerate}
\item Find $a_i,b_i,c_i$ for $1 \leq i \leq 3$.
\item Find basis of $\Image{T}$ and $\Ker{T}$.
\end{enumerate}
\subsection{Problem 6}
\label{sec-1-6}
\begin{enumerate}
\item Let $V$ be a finitely generated vector space and $T:V\to V$ a linear 
transformation.  Prove that if $T$ is not an isomorphism, then there
exists a baisis $B$ in $V$ such that $[T]_B$ is a matrix with a column of
zeros.
\item Prove that if $A$ is singular, then $A$ is similar to a matrix with a
column of zeros.
\end{enumerate}
% Emacs 25.0.50.1 (Org mode 8.2.2)
\end{document}