% Created 2016-05-20 Fri 11:37
\documentclass[11pt]{article}
\usepackage[utf8]{inputenc}
\usepackage[T1]{fontenc}
\usepackage{fixltx2e}
\usepackage{graphicx}
\usepackage{longtable}
\usepackage{float}
\usepackage{wrapfig}
\usepackage{rotating}
\usepackage[normalem]{ulem}
\usepackage{amsmath}
\usepackage{textcomp}
\usepackage{marvosym}
\usepackage{wasysym}
\usepackage{amssymb}
\usepackage{capt-of}
\usepackage[hidelinks]{hyperref}
\tolerance=1000
\usepackage[utf8]{inputenc}
\usepackage[usenames,dvipsnames]{color}
\usepackage{a4wide}
\usepackage{commath}
\usepackage{amsmath}
\usepackage{marginnote}
\usepackage{enumerate}
\usepackage{listings}
\usepackage{color}
\usepackage{breqn}
\usepackage{flexisym}
\usepackage{mathstyle}
\hypersetup{urlcolor=blue}
\hypersetup{colorlinks,urlcolor=blue}
\setlength{\parskip}{16pt plus 2pt minus 2pt}
\definecolor{codebg}{rgb}{0.96,0.99,0.8}
\DeclareMathOperator{\Sp}{Sp}
\DeclareMathOperator{\cis}{cis}
\author{Oleg Sivokon}
\date{\textit{<2016-05-20 Fri>}}
\title{Assignment 14, Linear Algebra 1}
\hypersetup{
  pdfkeywords={Assignment, Linear Algebra},
  pdfsubject={Third asssignment in the course Linear Algebra 1},
  pdfcreator={Emacs 25.1.50.2 (Org mode 8.2.10)}}
\begin{document}

\maketitle
\tableofcontents


\definecolor{codebg}{rgb}{0.96,0.99,0.8}
\lstnewenvironment{maxima}{%
  \lstset{backgroundcolor=\color{codebg},
    frame=single,
    framerule=0pt,
    basicstyle=\ttfamily\scriptsize,
    columns=fixed}}{}
}
\makeatletter
\newcommand{\verbatimfont}[1]{\renewcommand{\verbatim@font}{\ttfamily#1}}
\makeatother
\verbatimfont{\small}%
\makeatletter
\renewcommand*\env@matrix[1][*\c@MaxMatrixCols c]{%
  \hskip -\arraycolsep
  \let\@ifnextchar\new@ifnextchar
  \array{#1}}
\makeatother
\clearpage

\section{Problems}
\label{sec-1}

\subsection{Problem 1}
\label{sec-1-1}
Given $f, g, h$ are functions from $\mathbb{R}$ to $\mathbb{R}$, check that
all of them are linearly independent when:
\begin{enumerate}
\item $f(x) = \sin x$, $g(x) = \cos x$, $h(x) = x \cos x$.
\item $f(x) = x(x - 1)$, $g(x) = x(x - 2)$, $h(x) = (x - 1)(x - 2)$.
\item $f(x) = \sin^2 x$, $g(x) = \cos^2 x$, $h(x) = 3$.
\end{enumerate}

\subsubsection{Answer 1}
\label{sec-1-1-1}

\subsubsection{Answer 2}
\label{sec-1-1-2}

\subsubsection{Answer 3}
\label{sec-1-1-3}

\subsection{Problem 2}
\label{sec-1-2}
Given the following subsets of $\mathbb{R}^4$:
\begin{align*}
  U &= \{(x, y, z, t) \in \mathbb{R}^4 \;|\; x - y + z = 0 \land x - y - 2t = 0\} \\
  W &= \Sp\{(1,0,1,1), (0,1,0,-1), (1,0,1,0)\}
\end{align*}


\begin{enumerate}
\item Prove that $U$ and $W$ are subspaces of $\mathbb{R}^4$.
\item Find basis for $U$, $W$ and $U+W$.
\item Find basis for $U \cap W$.
\item Find subspace $T$ of $\mathbb{R}^4$ s.t. $U \oplus T = \mathbb{R}^4$.
\end{enumerate}

\subsubsection{Answer 4}
\label{sec-1-2-1}

\subsubsection{Answer 5}
\label{sec-1-2-2}

\subsubsection{Answer 6}
\label{sec-1-2-3}

\subsubsection{Answer 7}
\label{sec-1-2-4}

\subsection{Problem 3}
\label{sec-1-3}
Let $\vec{v}_1, \vec{v}_2, \dots, \vec{v}_k$ and $\vec{w}$ be vectors in
linear space $V$.  Given $\{\vec{v}_1, \vec{v}_2, \dots, \vec{v}_k\}$ is
linearly independent and that $\vec{w} \not \in \Sp\{\vec{v}_1, \vec{v}_2,
   \dots, \vec{v}_k\}$, prove that $\vec{v_1} \not \in \Sp\{\vec{v}_1 + \vec{w},
   \vec{v}_2 + \vec{w}, \dots, \vec{v}_k + \vec{w}\}$.

\subsubsection{Answer 8}
\label{sec-1-3-1}

\subsection{Problem 4}
\label{sec-1-4}
Let $U$ and $W$ be distinct linear subspaces of $\mathbb{R}^5$ of
dimension 3.  Suppose $(2, 1, 0, 1), (1, 0, 1, 1) \in U \cap W$, what is the
dimension of $U + W$?

\subsubsection{Answer 9}
\label{sec-1-4-1}

\subsection{Problem 5}
\label{sec-1-5}
Let $A$ and $B$ be square matrices of size $n$, $n \geq 2$.  Suppose $A$ and
$B$ are of the rank 1, 
\begin{enumerate}
\item what are the possible ranks of $A + B$?
\item What is the possible rank of $A + B$ when they both are of rank 2?
\item Prove that it is possible to write any matrix of rank 2 as a sum of
matrices of rank 1.
\end{enumerate}

\subsubsection{Answer 10}
\label{sec-1-5-1}

\subsection{Problem 6}
\label{sec-1-6}
Given bases $B = (\vec{u}_1, \vec{u}_2, \vec{u}_3)$ and $C = (\vec{v}_1,
   \vec{v}_2, \vec{v}_3)$ both in $\mathbb{R}^3$ s.t.
\begin{align*}
  \vec{u}_1 &= (2,1,1) \\
  \vec{u}_2 &= (2,-1,1) \\
  \vec{u}_3 &= (1,2,1) \\
  \vec{v}_1 &= (3,1,-5) \\
  \vec{v}_2 &= (1,1,-3) \\
  \vec{v}_3 &= (-1,0,2)
\end{align*}

\begin{enumerate}
\item Write the matrix of change of basis from $B$ to $C$ and its inverse.
\item Compute the coordinate vector $[w]_B$ where $\vec{w} = (-5,8,-5)$.
\item Similarly, compute $[w]_C$.
\end{enumerate}

\subsubsection{Answer 11}
\label{sec-1-6-1}

\subsubsection{Answer 12}
\label{sec-1-6-2}

\subsubsection{Answer 13}
\label{sec-1-6-3}
% Emacs 25.1.50.2 (Org mode 8.2.10)
\end{document}