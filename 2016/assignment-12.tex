% Created 2016-03-26 Sat 22:36
\documentclass[11pt]{article}
\usepackage[utf8]{inputenc}
\usepackage[T1]{fontenc}
\usepackage{fixltx2e}
\usepackage{graphicx}
\usepackage{longtable}
\usepackage{float}
\usepackage{wrapfig}
\usepackage{rotating}
\usepackage[normalem]{ulem}
\usepackage{amsmath}
\usepackage{textcomp}
\usepackage{marvosym}
\usepackage{wasysym}
\usepackage{amssymb}
\usepackage{capt-of}
\usepackage[hidelinks]{hyperref}
\tolerance=1000
\usepackage[utf8]{inputenc}
\usepackage[usenames,dvipsnames]{color}
\usepackage{a4wide}
\usepackage{commath}
\usepackage{amsmath}
\usepackage{marginnote}
\usepackage{enumerate}
\usepackage{listings}
\usepackage{color}
\hypersetup{urlcolor=blue}
\hypersetup{colorlinks,urlcolor=blue}
\setlength{\parskip}{16pt plus 2pt minus 2pt}
\definecolor{codebg}{rgb}{0.96,0.99,0.8}
\author{Oleg Sivokon}
\date{\textit{<2016-03-26 Sat>}}
\title{Assignment 12, Linear Algebra 1}
\hypersetup{
 pdfauthor={Oleg Sivokon},
 pdftitle={Assignment 12, Linear Algebra 1},
 pdfkeywords={Assignment, Linear Algebra},
 pdfsubject={Second asssignment in the course Linear Algebra 1},
 pdfcreator={Emacs 25.0.50.1 (Org mode 8.3beta)}, 
 pdflang={English}}
\begin{document}

\maketitle
\tableofcontents

\definecolor{codebg}{rgb}{0.96,0.99,0.8}
\lstnewenvironment{maxima}{%
  \lstset{backgroundcolor=\color{codebg},
    frame=single,
    framerule=0pt,
    basicstyle=\ttfamily\scriptsize,
    columns=fixed}}{}
}
\makeatletter
\newcommand{\verbatimfont}[1]{\renewcommand{\verbatim@font}{\ttfamily#1}}
\makeatother
\verbatimfont{\small}%
\makeatletter
\renewcommand*\env@matrix[1][*\c@MaxMatrixCols c]{%
  \hskip -\arraycolsep
  \let\@ifnextchar\new@ifnextchar
  \array{#1}}
\makeatother
\clearpage

\section{Problems}
\label{sec:orgheadline12}

\subsection{Problem 1}
\label{sec:orgheadline3}
Let \(A\) be a square matrix of order \(3 \times 3\) s.t. \(A^3 = 0\), but
\(A^2 \neq 0\).

\begin{enumerate}
\item Prove that there exists a vector \(\vec{v} \in \mathbb{R}^3\) s.t. \(A\vec{v}
      \neq 0\).
\item Prove that there exits vectors \(\vec{v} \in \mathbb{R}^3\) s.t.
\(\{\vec{v}, \vec{v}A, \vec{v}A^2\}\) are the basis of \(\mathbb{R}^3\).
\end{enumerate}

\subsubsection{Answer 1}
\label{sec:orgheadline1}
Notice that \(A\) itself is made of the column vectors, call them \(r_1, r_2,
    r_3\).  All of which are in \(\mathbb{R}^3\).  Suppose, for contradiction, that
there is no vector in \(\vec{v} \in \mathbb{R}^3\) satisfying \(A\vec{v} \neq
    0\).  In particular, none of the \(\vec{r}_1, \vec{r}_2, \vec{r}_3\) satisfies
the above condition.  In other words, \(A\vec{r}_1 = 0\), \(A\vec{r}_2 = 0\),
\(A\vec{r}_3 = 0\).  (where 0 means zero matrix).  On the other hand,
\(A\vec{r}_1 + A\vec{r}_2 + A\vec{r}_3 = A^2 \neq 0\).  Contradiction.  Hence,
there exists \(\vec{v} \in \mathbb{R}^3\) s.t. \(A\vec{v} \neq 0\).

\subsubsection{Answer 2}
\label{sec:orgheadline2}

\subsection{Problem 2}
\label{sec:orgheadline5}
Given square matrices \(A, B, C, D\) of order \(n \times n\) s.t. \(ABCD = I\),
prove that \(ABCD = DABC = CDAB = BCDA = I\).

\subsubsection{Answer 3}
\label{sec:orgheadline4}
The proof is immediate from the definition of inverse: \(XX^{-1} = I\) and
associativity of matrix multiplication.  In other words:

\begin{align*}
  ABCD &= I \iff \\
  A(BCD) &= I \iff \\
  A^{-1} &= BCD \iff \\
  DABC &= I \iff \\
  AB(CD) &= I \iff \\
  (AB)^{-1} &= CD \iff \\
  CDAB &= I \iff \\
  A^{-1}A &= I \iff \\
  I &= BCDA \;.
\end{align*}

\subsection{Problem 3}
\label{sec:orgheadline7}
Let \(A\) be a square matrix of order \(m \times m\), let \(B\) be a matrix of
order \(m \times n\).  Prove in two different ways that if \(A\) is invertible,
then \(B\vec{x} = 0\) and \(AB\vec{x} = 0\) has the same solution space.

\subsubsection{Answer 4}
\label{sec:orgheadline6}

\subsection{Problem 4}
\label{sec:orgheadline9}
Given matrix \(A\) of a general form:

\begin{align*}
  \begin{bmatrix}
    0      & a_1    & \dots  & 0      & 0 \\
    0      & 0      & a_2    & \dots  & 0 \\
    \vdots & \vdots & \vdots & \ddots & \vdots \\
    0      & 0      & 0      & \dots  & a_{n-1} \\
    a_n    & 0      & 0      & \dots  & 0
  \end{bmatrix}
\end{align*}

Prove that it is invertible, show \(A^{-1}\).

\subsubsection{Answer 5}
\label{sec:orgheadline8}
Performing elementary operations: \(R_n \to R_1\) and \(R_k \to R_{k+1}, 1 \leq
    k < n\) gives us diagonal matrix.  This matrix is invertible since it has a
pivot element in each of its columns.

The inverse of \(A\) will, in general look like this:

\begin{align*}
  \begin{bmatrix}
    0             & 0             & \dots  & 0                & \frac{1}{a_n} \\
    \frac{1}{a_1} & 0             & \dots  & 0                & 0 \\
    0             & \frac{1}{a_2} & \dots  & 0                & 0 \\
    \vdots        & \vdots        & \ddots & \vdots           & \vdots \\
    0             & 0             & \dots  & \frac{1}{a_{n-1}} & 0 \\
  \end{bmatrix}
\end{align*}

Notice that for each row of \(A\), we will be matching the column of \(A^{-1}\).
We need to make sure that the only non-zero element of \(A_c\) was matched by
the only non-zero element of \(A^{-1}_r\) (where \(c\) stands for column index
and \(r\) stands for row index).  In order to obtain a diagonal with all ones
(i.e. the identity matrix), we need to also make sure that \(A_{c,i} \times
    A^{-1}_{r,j} = 1\).  In other words, we need to match \(a_1\) with \(\frac{1}{a_1}\),
\(a_2\) with \(\frac{1}{a_2}\), and so on.

\subsection{Problem 5}
\label{sec:orgheadline11}
Let \(A\) and \(B\) be square matrices of the order \(3 \times 3\) s.t. \(B^2A = -2B^3\)
and \(B^3 + AB^2 = 3I\).

Prove that \(A\) and \(B\) are invertible and express \(A^{-1}\) and \(B^{-1}\) in
terms of \(B\).

\subsubsection{Answer 6}
\label{sec:orgheadline10}
Using some matrix algebra we obtain: \(B^{-1} = -\frac{1}{3}B^2\) and \(A^{-1}
    = (-2B)^{-1}\).

\begin{align*}
  B^2A &= -2B^3 \iff \\
  B^2A &= B^2(-2I)B \iff \\
  A &= -2B \\
  &\textit{substituting into second equation:} \\
  B^3 + AB^2 &= 3I \iff \\
  B^3 - 2B^3 &= 3I \iff \\
  -B^3 &= 3I \iff \\
  B(-B^2) &= 3I \iff \\
  B(-\frac{1}{3}B^2) &= I \iff \\
  B^{-1} &= -\frac{1}{3}B^2 \\
  &\textit{$A$ is invertible because it is similar to $B$} \\
  -2B &= (\sqrt{2}I)B(\sqrt{2}I^{-1}) \\
  A^{-1} &= (-2B)^{-1} \;.
\end{align*}
\end{document}