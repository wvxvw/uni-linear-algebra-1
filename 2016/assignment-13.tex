% Created 2016-05-14 Sat 14:38
\documentclass[11pt]{article}
\usepackage[utf8]{inputenc}
\usepackage[T1]{fontenc}
\usepackage{fixltx2e}
\usepackage{graphicx}
\usepackage{longtable}
\usepackage{float}
\usepackage{wrapfig}
\usepackage{rotating}
\usepackage[normalem]{ulem}
\usepackage{amsmath}
\usepackage{textcomp}
\usepackage{marvosym}
\usepackage{wasysym}
\usepackage{amssymb}
\usepackage{capt-of}
\usepackage[hidelinks]{hyperref}
\tolerance=1000
\usepackage[utf8]{inputenc}
\usepackage[usenames,dvipsnames]{color}
\usepackage{a4wide}
\usepackage{commath}
\usepackage{amsmath}
\usepackage{marginnote}
\usepackage{enumerate}
\usepackage{listings}
\usepackage{color}
\usepackage{breqn}
\usepackage{flexisym}
\usepackage{mathstyle}
\hypersetup{urlcolor=blue}
\hypersetup{colorlinks,urlcolor=blue}
\setlength{\parskip}{16pt plus 2pt minus 2pt}
\definecolor{codebg}{rgb}{0.96,0.99,0.8}
\DeclareMathOperator{\Sp}{Sp}
\DeclareMathOperator{\cis}{cis}
\author{Oleg Sivokon}
\date{\textit{<2016-04-16 Sat>}}
\title{Assignment 13, Linear Algebra 1}
\hypersetup{
  pdfkeywords={Assignment, Linear Algebra},
  pdfsubject={Third asssignment in the course Linear Algebra 1},
  pdfcreator={Emacs 25.1.50.2 (Org mode 8.2.10)}}
\begin{document}

\maketitle
\tableofcontents


\definecolor{codebg}{rgb}{0.96,0.99,0.8}
\lstnewenvironment{maxima}{%
  \lstset{backgroundcolor=\color{codebg},
    frame=single,
    framerule=0pt,
    basicstyle=\ttfamily\scriptsize,
    columns=fixed}}{}
}
\makeatletter
\newcommand{\verbatimfont}[1]{\renewcommand{\verbatim@font}{\ttfamily#1}}
\makeatother
\verbatimfont{\small}%
\makeatletter
\renewcommand*\env@matrix[1][*\c@MaxMatrixCols c]{%
  \hskip -\arraycolsep
  \let\@ifnextchar\new@ifnextchar
  \array{#1}}
\makeatother
\clearpage

\section{Problems}
\label{sec-1}

\subsection{Problem 1}
\label{sec-1-1}
\begin{enumerate}
\item Find all solutions of $z^3 + 3i\overline{z} = 0$.
\item Let $z_1, z_2$ be complex numbers.  Prove that unless $z_1z_2 = 1$ and
$\abs{z_1} = \abs{z_2} = 1$, then $\frac{z_1 + z_2}{1 + z_1z_2}$ is a real
number.
\end{enumerate}

\subsubsection{Answer 1}
\label{sec-1-1-1}
First, note that zero is a solution of this equation.  Other roots can be
found as follows:

\begin{align*}
  z &=  r\cis \theta \\
  \\
  r^3 \cis(3 \theta) + 3i \cis(-\theta) &= 0 \iff \\
  r^3 \cis(3 \theta) + 3(i \cos(-\theta) + i^2 \sin(-\theta)) &= 0 \iff \\
  r^3 \cis(3 \theta) + 3(i \sin(\theta) + \cos(\theta)) &= 0 \iff \\
  r^3 \cis(3 \theta) + 3 \cis(\theta) &= 0 \iff \\
  r^3 \cis(3 \theta) &= -3 \cis(\theta) \\
  \textit{Equating radius and angle:} & \\
  r^3 &= -3 \iff \\
  r &= -3^{\frac{1}{3}} \\
  3 \theta &= \theta \mod 2\pi \iff \\
  2 \theta &= 0 \mod 2\pi \iff \\
  \theta &= \pi \mod 2\pi \;.
\end{align*}

Hence, other roots are:

\begin{align*}
  z_1 &= -3^{\frac{1}{3}}\cis(0) \\
  z_2 &= -3^{\frac{1}{3}}\cis(\pi)
\end{align*}

\subsubsection{Answer 3}
\label{sec-1-1-2}
Since $\abs{z_1}^2 = z_1 \overline{z_1} = 1$, we have that $z_1 =
    \overline{z_1} = \frac{1}{z_1}$, and similarly $z_2 = \overline{z_2} =
    \frac{1}{z_2}$.  Further using the properties of complex conjugate we have:

\begin{align*}
  \frac{z_1 + z_2}{1 + z_1z_2} &= \frac{\frac{1}{z_1} + \frac{1}{z_2}}{1 + \frac{1}{z_1z_2}} \\
                               &= \frac{\overline{z_1} + \overline{z_2}}{1 + \overline{z_1z_2}} \\
                               &= \overline{\frac{z_1 + z_2}{1 + z_1z_2}}
\end{align*}

And since $z = \overline{z} \implies z \in \mathbb{R}$, we have that the given
expression is real.

\subsection{Problem 2}
\label{sec-1-2}
Let $\mathbb{Q}$ denote the field of rational numbers.  And $K$ defined as follows:
\begin{align*}
K = \left\{\begin{bmatrix}
      a & 2b \\
      b & a
    \end{bmatrix} \;|\; a,b \in \mathbb{Q}\right\}\;.
\end{align*}

Is $K$ a field under matrix addition and multiplication?

\subsubsection{Answer 3}
\label{sec-1-2-1}
Yes, $K$ is a field, following is the illustration of field axioms satisfied
by $K$.

\begin{enumerate}
\item Closure under addition:
\begin{align*}
  \begin{bmatrix}
    a & 2b \\
    b & a
  \end{bmatrix} + 
  \begin{bmatrix}
    c & 2d \\
    d & c
  \end{bmatrix} =
  \begin{bmatrix}
    a + c & 2(b + d) \\
    b + d & a + c
  \end{bmatrix}
\end{align*}

where $(a + c) \in \mathbb{Q} = e$, $(b + d) \in \mathbb{Q} = f$, results in
a general matrix:
\begin{align*}
  \begin{bmatrix}
    e & 2f \\
    f & e
  \end{bmatrix} \in K
\end{align*}

\item Closure under multiplication:
\begin{align*}
  \begin{bmatrix}
    a & 2b \\
    b & a
  \end{bmatrix} \times 
  \begin{bmatrix}
    c & 2d \\
    d & c
  \end{bmatrix} =
  \begin{bmatrix}
    ac + 2db & 2ad + 2bc \\
    bc + ad & 2db + ac
  \end{bmatrix}
\end{align*}

where $(ac + 2db) \in \mathbb{Q} = e$ and $2(ad + 2bc) \in \mathbb{Q} =
       f$ and, similarly:
\begin{align*}
  \begin{bmatrix}
    e & 2f \\
    f & e
  \end{bmatrix} \in K
\end{align*}

\item Associativity of addition:
\begin{align*}
  \left(\begin{bmatrix}
    a & 2b \\
    b & a
  \end{bmatrix} + 
  \begin{bmatrix}
    c & 2d \\
    d & c
  \end{bmatrix}\right) +
  \begin{bmatrix}
    e & 2f \\
    f & e
  \end{bmatrix} =
  \begin{bmatrix}
    a + c & 2(b + d) \\
    b + d & a + c
  \end{bmatrix} +
  \begin{bmatrix}
    e & 2f \\
    f & e
  \end{bmatrix} = \\
  \begin{bmatrix}
    a + c + e & 2(b + d + f) \\
    b + d + f & a + c + e
  \end{bmatrix} = \\
  \begin{bmatrix}
    a & 2b \\
    b & a
  \end{bmatrix} + 
  \begin{bmatrix}
    c + e & 2(d + f) \\
    d + f & c + e
  \end{bmatrix} =
  \begin{bmatrix}
    a & 2b \\
    b & a
  \end{bmatrix} + 
  \left(\begin{bmatrix}
    c & 2d \\
    d & c
  \end{bmatrix} +
  \begin{bmatrix}
    e & 2f \\
    f & e
  \end{bmatrix}\right)
\end{align*}

\item Associativity of multiplication:
\begin{align*}
  \left(\begin{bmatrix}
    a & 2b \\
    b & a
  \end{bmatrix} \times 
  \begin{bmatrix}
    c & 2d \\
    d & c
  \end{bmatrix}\right) \times 
  \begin{bmatrix}
    e & 2f \\
    f & e
  \end{bmatrix} =
  \begin{bmatrix}
    ac + 2db & 2ad + 2bc \\
    bc + ad & 2db + ac
  \end{bmatrix} \times 
  \begin{bmatrix}
    c & 2d \\
    d & c
  \end{bmatrix} = \\
  \begin{bmatrix}
    e(ac + 2db) + f(2ad + 2bc) & 2f(ad + 2db) + e(2ad + 2bc) \\
    e(bc + ad) + f(2db + ac) & 2f(bc + ad) + e(2db + ac)
  \end{bmatrix} = \\
  \begin{bmatrix}
    a & 2b \\
    b & a
  \end{bmatrix} \times
  \begin{bmatrix}
    ec + 2df & 2fc + 2ed \\
    ed + fc & 2fd + ec
  \end{bmatrix} =
  \begin{bmatrix}
    a & 2b \\
    b & a
  \end{bmatrix} \times 
  \left(\begin{bmatrix}
    c & 2d \\
    d & c
  \end{bmatrix} \times 
  \begin{bmatrix}
    e & 2f \\
    f & e
  \end{bmatrix}\right)
\end{align*}

\item Commutativity of addition:
\begin{align*}
  \begin{bmatrix}
    a & 2b \\
    b & a
  \end{bmatrix} + 
  \begin{bmatrix}
    c & 2d \\
    d & c
  \end{bmatrix} =
  \begin{bmatrix}
    a + c & 2(b + d) \\
    b + d & a + c
  \end{bmatrix} = 
  \begin{bmatrix}
    c & 2d \\
    d & c
  \end{bmatrix} + 
  \begin{bmatrix}
    a & 2b \\
    b & a
  \end{bmatrix}
\end{align*}

\item Commutativity of multiplication:
\begin{align*}
  \begin{bmatrix}
    a & 2b \\
    b & a
  \end{bmatrix} \times
  \begin{bmatrix}
    c & 2d \\
    d & c
  \end{bmatrix} =
  \begin{bmatrix}
    ac + 2db & 2ad + 2bc \\
    bc + ad & 2db + ac
  \end{bmatrix} = 
  \begin{bmatrix}
    c & 2d \\
    d & c
  \end{bmatrix} \times
  \begin{bmatrix}
    a & 2b \\
    b & a
  \end{bmatrix}
\end{align*}

\item Additive identity is the zero matrix (from matrix addition properties).
\item Multiplicative identity is the identity matrix (again, from matrix
multiplication properties).
\item General inverse under addition:
\begin{align*}
  \left(\begin{bmatrix}
    a & 2b \\
    b & a
  \end{bmatrix}\right)^{-1} =
  \begin{bmatrix}
    c & 2d \\
    d & c
  \end{bmatrix} \\
  \textit{where } c + a = 0, d + b = 0 \\
  \textit{i.e. } c = -a, d = -b \\
  \left(\begin{bmatrix}
    a & 2b \\
    b & a
  \end{bmatrix}\right)^{-1} =
  \begin{bmatrix}
    -a & -2b \\
    -b & -a
  \end{bmatrix}
\end{align*}

\item General inverse under multiplication.  First, we will find the
determinant of a generic matrix in $K$:
\begin{align*}
D = \left|
  \begin{array}{ll}
    a & 2b \\
    b & a
  \end{array} \right| = aa - 2bb = a^2 - 2b^2
\end{align*}

Since 2 appears without a pair in the expression $2b^2$, it means that
the prime factorization of this expression contains an odd number of
twos.  Hence, it is not possible for $a^2$ to be equal to $2b^2$, unless
both $a = 0$ and $b = 0$.  Hence, the only element of $K$ which doesn't
have an inverse is the zero matrix.  For every other element its inverse
is:
\begin{align*}
  \left(\begin{bmatrix}
    a & 2b \\
    b & a 
  \end{bmatrix}\right)^{-1} =
  \frac{1}{D}\begin{bmatrix}
    a  & -2b \\
    -b & a 
  \end{bmatrix} = 
  \frac{1}{a^2 - 2b^2}\begin{bmatrix}
    a  & -2b \\
    -b & a 
  \end{bmatrix} = 
  \begin{bmatrix}
    \frac{a}{a^2 - 2b^2}  & \frac{-2b}{a^2 - 2b^2} \\
    \frac{-b}{a^2 - 2b^2} & \frac{a}{a^2 - 2b^2}
  \end{bmatrix}
\end{align*}

As before, $\frac{a}{a^2 + 2b^2} \in \mathbb{Q} = e$ and $\frac{-b}{a^2 +
        2b^2} \in \mathbb{Q} = f$, hence:
\begin{align*}
  \begin{bmatrix}
    e & 2f \\
    f & e
  \end{bmatrix} \in K
\end{align*}

\item Finally, distributivity of multiplication over addition:
\begin{align*}
  \begin{bmatrix}
    a & 2b \\
    b & a
  \end{bmatrix} \times 
  \left(\begin{bmatrix}
    c & 2d \\
    d & c
  \end{bmatrix} + 
  \begin{bmatrix}
    e & 2f \\
    f & e
  \end{bmatrix}\right) =
  \begin{bmatrix}
    a & 2b \\
    b & a
  \end{bmatrix} \times 
  \begin{bmatrix}
    c + e & 2(d + f) \\
    d + f & c + e 
  \end{bmatrix} = \\
  \begin{bmatrix}
    a(c + e) + 2b(d + f) & 2b(c + e) + 2a(d + f) \\
    a(d + f) + b(c + e)  & 2b(d + f) + a(c + e)
  \end{bmatrix} = \\
  \begin{bmatrix}
    ac + 2bd & 2ad + 2bc \\
    bc + ad & 2bd + ac
  \end{bmatrix} + 
  \begin{bmatrix}
    ae + 2bf & 2af + 2be \\
    be + af & 2bf + ae
  \end{bmatrix} = \\  
  \begin{bmatrix}
    a & 2b \\
    b & a
  \end{bmatrix} \times 
  \begin{bmatrix}
    c & 2d \\
    d & c
  \end{bmatrix} + 
  \begin{bmatrix}
    a & 2b \\
    b & a
  \end{bmatrix} \times 
  \begin{bmatrix}
    e & 2f \\
    f & e
  \end{bmatrix}
\end{align*}
\end{enumerate}

\subsection{Problem 3}
\label{sec-1-3}
Verify that $V$ is a vectors space over field $\textbf{F}$:
\begin{enumerate}
\item $\textbf{F} = \mathbb{C}, V = \mathbb{M}_{n\times n}^{\mathbb{C}}$ and
addition defined to be the regular addition, while multiplication is
defined to be $\lambda\times A = \abs{\lambda}\times A$.
\item $\textbf{F} = \mathbb{R}, V = \{(x,y) \in \mathbb{R}^2\;|\;y=2x\}$,
with addition being the addition of $\mathbb{R}^2$ and multiplication
$\lambda\times (x,y) = (\lambda x, 0)$.
\end{enumerate}

\subsubsection{Answer 5}
\label{sec-1-3-1}
\begin{enumerate}
\item Distributivity of scalar addition prevents $V$ from being a field over
$\textbf{F}$.  Consider this example:
\begin{align*}
  \abs{-1+i} \times
  \begin{bmatrix}
    1 & 0 \\
    0 & 1
  \end{bmatrix} +
  \abs{1-i} \times
  \begin{bmatrix}
    1 & 0 \\
    0 & 1
  \end{bmatrix} &= \\
  \sqrt{(-1)^2+1^2} \times
  \begin{bmatrix}
    1 & 0 \\
    0 & 1
  \end{bmatrix} +
  \sqrt{1^2+(-1)^2} \times
  \begin{bmatrix}
    1 & 0 \\
    0 & 1
  \end{bmatrix} &=
  \begin{bmatrix}
    2\sqrt{2} & 0 \\
    0 & 2\sqrt{2}
  \end{bmatrix}
\end{align*}

while, at the same time:
\begin{align*}
  \abs{(-1+i) + (1-i)} \times
  \begin{bmatrix}
    1 & 0 \\
    0 & 1
  \end{bmatrix} = 
  \sqrt{0^2+0^2} \times
  \begin{bmatrix}
    1 & 0 \\
    0 & 1
  \end{bmatrix} &=
  \begin{bmatrix}
    0 & 0 \\
    0 & 0
  \end{bmatrix}
\end{align*}

Obviously, 
\begin{align*}
  \begin{bmatrix}
    2\sqrt{2} & 0 \\
    0 & 2\sqrt{2}
  \end{bmatrix} \neq 
  \begin{bmatrix}
    0 & 0 \\
    0 & 0
  \end{bmatrix}
\end{align*}

\item Of course $V$ is not a vector space over $\mathbf{F}$, almost none of
scalar multiples are in $V$, since they are of the form $(x, y)$, where
$y = 0$ and $y = 2x$, but this is only true when $x = 0$ as well.  Any
other value of $x$ will not satisfy closure under multiplication
requirement.
\end{enumerate}

\subsection{Problem 4}
\label{sec-1-4}
Given sets:
\begin{enumerate}
\item $K = \{(x,y,z,t) \in \mathbb{R}^4\;|\; x+y-z+t=0 \land 2x+y+z-3t=0\}$.
\item $L = \{f \in V\;|\; f\left(\frac{1}{2}\right)>f(2)\}$, $V$ is the vector
space of all real-valued functions.
\item $M = \{p(x) \in \mathbb{R}^4[x]\;|\; p(-3) = 0\}$.
\item $R = \{(x,y) \in \mathbb{R}^3\;|\; x^2 + y^2 = 0\}$.
\item $R = \{(x,y) \in \mathbb{R}^3\;|\; x^2 - y^2 = 0\}$.
\end{enumerate}

Fore each of sets given, assert them being vector spaces.  In case they are,
prove this in two different ways.

\subsubsection{Answer 6}
\label{sec-1-4-1}
\begin{enumerate}
\item By substitution find that $x = 4t - 2z$, $y = 3z - 5t$.  This gives us
vectors $\vec{v}_1 = (4, -5, 0, 1)^T$ and $\vec{v}_2 = (-2, 3, 1, 0)^T$
which span $K$.  In other words, $K$ is a vector space over the field of
real numbers with the operations of vector addition and multiplication.
\item $L$ is not a vector space.  For example, it doesn't have additive
inverse.  Suppose for contradiction that there exists additive inverse in
$L$, then $f(x) + f'(x) = 0$, in particular, $f(\frac{1}{2}) +
       f'(\frac{1}{2}) = 0$ and $f(2) + f'(2) = 0$.  We know that
$f(\frac{1}{2}) > f(2)$.  Let $f(\frac{1}{2}) = n$ and $f(2) = m$.  Then
$f'(\frac{1}{2}) = -n$ and $f'(2) = -m$.  However, if $n > m$, then $-n <
       -m$.  Contradiction.  Hence, $L$ is not a vector space.
\item $M$ is a vector space with the span $P = \Sp\{(1, 0, 0, (-3)^3), (0, 1,
       0, (-3)^2), (0, 0, 1, (-3)^1), \allowbreak (0, 0, 0, 0)\}$.  To see why
these vectors span $M$ suppose there was a polinomial $p(x)$ s.t. $p(-3)
       = 0$, but it is not a linear combination of $P$.  However, $p(x)$ must be
representable as follows $(\alpha(-3)^3 - c_{\alpha}) + (\beta(-3)^2 -
       c_{\beta}) + (\gamma(-3)^1 - c_{\gamma}) = 0$, with $c_i$ some constants.
Now, note that each of the summands individually can be represented by
the elements of $P$, hence, contrary to assumed, $p(x)$ is a linear
combinaton of $P$.  Hence, $P$ spans $M$.
\item $R$ is a vector space, if you allow vector spaces with just one element:
the condition $x^2+y^2=0$ in real numbers can only be satisfied when
$x=y=0$, since squares of real numbers are non-negative.  This space
would be the $\Sp\{(0, 0)\}$.
\item $S$ is a vector space defined by $\Sp\{(1, 1), (0, 0)\}$, it is equivalent
to just the real numbers.
\end{enumerate}

\subsection{Problem 5}
\label{sec-1-5}
Given vector space $V$ and $\vec{v}_1, \vec{v}_2, \vec{v}_3$ distinct vectors
prove or disprove:
\begin{enumerate}
\item If $\Sp\{\vec{v}_1, \vec{v}_2\} = \Sp\{\vec{v}_1, \vec{v}_3\}$, then
$\vec{v}_2$ is a multiple of $\vec{v}_3$.
\item If $\vec{v}_1 - 2\vec{v}_2 + \vec{v}_3 = \vec{0}$, then 
$\Sp\{\vec{v}_1, \vec{v}_2\} = \Sp\{\vec{v}_1, \vec{v}_3\}$.
\item If $\{\vec{v}_1, \vec{v}_2, \vec{v}_3\}$ is linearly dependent, then
$\Sp\{\vec{v}_1, \vec{v}_2\} = \Sp\{\vec{v}_1 + \vec{v}_3, \vec{v}_2 +
      \vec{v}_3\}$.
\end{enumerate}

\subsubsection{Answer 7}
\label{sec-1-5-1}
\begin{enumerate}
\item False, counterexample: $\vec{v}_1 = (1, 0)^T$, $\vec{v}_2 = (1, 1)^T$,
$\vec{v}_3 = (0, 1)^T$, but there doesn't exist $\lambda$ s.t.
$\lambda \vec{v}_2 = \vec{v}_3$.
\item True, take any vector generated by the first span: 
\begin{align*}
  \vec{x} &= \alpha(2\vec{v}_2 - \vec{v}_3) + \beta \vec{v}_2 \\
  \vec{y} &= \gamma(2\vec{v}_2 - \vec{v}_3) + \delta \vec{v}_3 \\
  &\textit{Group the coefficients: } \\
  \vec{x} &= (2\alpha + \beta) \vec{v}_2 - \alpha \vec{v}_3 \\
  \vec{y} &= 2\gamma \vec{v}_2 - (\delta - \gamma) \vec{v}_3
\end{align*}

Since both $\vec{x}$ and $\vec{y}$ are linear combinations of $\vec{v}_2$
and $\vec{v}_3$, they are in the same span.  Hence $\Sp\{\vec{v}_1,
       \vec{v}_2\} = \Sp\{\vec{v}_1, \vec{v}_3\}$.
\item False, counterexample: $\vec{v}_1 = (0, 0)^T$, $\vec{v}_2 = (1, 0)^T$,
$\vec{v}_3 = (0, 1)^T$, but $\dim\Sp\{\vec{v}_1, \vec{v}_2\} = 1$ and
$\dim\Sp\{\vec{v}_2, \vec{v}_3\} = 2$.
\end{enumerate}

\subsection{Problem 6}
\label{sec-1-6}
Given following subspaces of $\mathbb{R}^3$:
$U = \Sp\{(1,1,2), (2,2,1)\}$ and $W = \Sp\{(1,3,4), (2,5,1)\}$,
find spanning set of $U \cap W$.

\subsubsection{Answer 8}
\label{sec-1-6-1}
Using linear space sum dimension theorem: $\dim(W + U) = \dim W + \dim U -
    \dim(W \cap U)$ we have that $\dim(W \cap U) = \dim W + \dim U - \dim(W +
    U)$.
Now, let's find the summands:
\begin{align*}
  \dim{U} &= \dim \left(\begin{bmatrix}
           1 & 1 & 2 \\
           2 & 2 & 1
         \end{bmatrix} \right) \\
          &= \dim \left(\begin{bmatrix}
           1 & 1 & 2 \\
           0 & 0 & -3
         \end{bmatrix} \right) \\
          &= 2\;. \\
  \dim{W} &= \dim \left(\begin{bmatrix}
           1 & 3 & 4 \\
           2 & 5 & 1
         \end{bmatrix} \right) \\
          &= \dim \left(\begin{bmatrix}
           1 & 1  & 2 \\
           0 & -1 & -7
         \end{bmatrix} \right) \\
          &= 2\;. \\
  \dim(W + U) &= \dim \left(\begin{bmatrix}
           1 & 3  & 4 \\
           0 & -1 & -7 \\
           1 & 1  & 2 \\
           0 & 0  & -3
         \end{bmatrix} \right) \\
          &= \dim \left(\begin{bmatrix}
           1 & 3  & 4 \\
           0 & -1 & -7 \\
           0 & -2 & 2 \\
           0 & 0  & -3
         \end{bmatrix} \right) \\
          &= \dim \left(\begin{bmatrix}
           1 & 3  & 4 \\
           0 & -1 & -7 \\
           0 & 0  & 16 \\
           0 & 0  & -3
         \end{bmatrix} \right) \\
          &= \dim \left(\begin{bmatrix}
           1 & 3  & 4 \\
           0 & -1 & -7 \\
           0 & 0  & 16 \\
           0 & 0  & 0
         \end{bmatrix} \right) \\
          &= 3\;.
\end{align*}

Imporatnt observation here is that the number of pivot elements in reduced
echelon form is the dimension of the matrix.

Hence, $\dim(W \cap U) = 2 + 2 - 3 = 1$, or, in other words, $U$ and $W$
share one common axis.
% Emacs 25.1.50.2 (Org mode 8.2.10)
\end{document}