% Created 2016-06-03 Fri 14:40
\documentclass[11pt]{article}
\usepackage[utf8]{inputenc}
\usepackage[T1]{fontenc}
\usepackage{fixltx2e}
\usepackage{graphicx}
\usepackage{longtable}
\usepackage{float}
\usepackage{wrapfig}
\usepackage{rotating}
\usepackage[normalem]{ulem}
\usepackage{amsmath}
\usepackage{textcomp}
\usepackage{marvosym}
\usepackage{wasysym}
\usepackage{amssymb}
\usepackage{capt-of}
\usepackage[hidelinks]{hyperref}
\tolerance=1000
\usepackage[utf8]{inputenc}
\usepackage[usenames,dvipsnames]{color}
\usepackage{a4wide}
\usepackage{commath}
\usepackage{amsmath}
\usepackage{marginnote}
\usepackage{enumerate}
\usepackage{listings}
\usepackage{color}
\usepackage{breqn}
\usepackage{flexisym}
\usepackage{mathstyle}
\hypersetup{urlcolor=blue}
\hypersetup{colorlinks,urlcolor=blue}
\setlength{\parskip}{16pt plus 2pt minus 2pt}
\definecolor{codebg}{rgb}{0.96,0.99,0.8}
\DeclareMathOperator{\Sp}{Sp}
\DeclareMathOperator{\image}{\mathrm{im}}
\author{Oleg Sivokon}
\date{\textit{<2016-06-03 Fri>}}
\title{Assignment 15, Linear Algebra 1}
\hypersetup{
  pdfkeywords={Assignment, Linear Algebra},
  pdfsubject={Third asssignment in the course Linear Algebra 1},
  pdfcreator={Emacs 25.1.50.2 (Org mode 8.2.10)}}
\begin{document}

\maketitle
\tableofcontents


\definecolor{codebg}{rgb}{0.96,0.99,0.8}
\lstnewenvironment{maxima}{%
  \lstset{backgroundcolor=\color{codebg},
    frame=single,
    framerule=0pt,
    basicstyle=\ttfamily\scriptsize,
    columns=fixed}}{}
}
\makeatletter
\newcommand{\verbatimfont}[1]{\renewcommand{\verbatim@font}{\ttfamily#1}}
\makeatother
\verbatimfont{\small}%
\makeatletter
\renewcommand*\env@matrix[1][*\c@MaxMatrixCols c]{%
  \hskip -\arraycolsep
  \let\@ifnextchar\new@ifnextchar
  \array{#1}}
\makeatother
\clearpage

\section{Problems}
\label{sec-1}

\subsection{Problem 1}
\label{sec-1-1}
For each of the given transformations check if it is linear:
\begin{enumerate}
\item $T:\mathbb{R}_2[x] \to \mathbb{R}_4[x]$ defined as $T(f(x)) = (x^3 -
      x)f(x^2)$.
\item $T:\mathbb{M}^{\mathbb{R}}_{n\times n} \to
      \mathbb{M}^{\mathbb{R}}_{n\times n}$ defined as $T(X) = AXA$ for some $A
      \in \mathbb{M}^{\mathbb{R}}_{n\times n}$.
\end{enumerate}

\subsubsection{Answer 1}
\label{sec-1-1-1}

\subsubsection{Answer 2}
\label{sec-1-1-2}

\subsection{Problem 2}
\label{sec-1-2}
\begin{enumerate}
\item Does there exist an isomorphism $T:\mathbb{R}_3[x] \to \mathbb{R}^3$ for
which $T(x^2 + 2x) = (1, 2, 1)$, $T(x + 1) = (0, 1, 1)$, $T(x^2 - 2) = (1,
      0, -1)$?
\item Given linear space $V$ and linear transformations $S, T: V \to V$, prove
that, if $V$ is finite-dimensional and $\ker S = \{0\}$, then $\image TS =
      \image S$.
\end{enumerate}

\subsubsection{Answer 3}
\label{sec-1-2-1}

\subsubsection{Answer 4}
\label{sec-1-2-2}

\subsection{Problem 3}
\label{sec-1-3}
Let $T: \mathbb{R}^5 \to \mathbb{R}^5$ be a linear transformation s.t. $T^2 =
   0$.
\begin{enumerate}
\item Prove that $\image T \subseteq \ker T$.
\item What are the possible values for the dimension of $\ker T$?
\item Let $U$ be a subspace of $\mathbb{R}^5$ s.t. $\dim U = 3$, prove that $U
      \cap \ker T \neq \{0\}$.
\end{enumerate}

\subsubsection{Answer 5}
\label{sec-1-3-1}

\subsubsection{Answer 6}
\label{sec-1-3-2}

\subsubsection{Answer 7}
\label{sec-1-3-3}

\subsection{Problem 4}
\label{sec-1-4}
Let $a \in \mathbb{R}$ and $T:\mathbb{R}^3 \to \mathbb{R}^3$ be a linear
transformation.  Let $B = ((1,0,0),(1,1,0),(1,1,1))$ be a basis in
$\mathbb{R}^3$.  Then $T$ with respect to the basis $B$ is given by
\begin{align*}
  [T]_B = \begin{bmatrix}
    a     & 1 - 1 & 0 \\
    a     & 2a    & 2a + 2 \\
    a + 1 & a + 1 & 2a + 2
  \end{bmatrix} \;.
\end{align*}

Also, $(2, 2, 2) \in \ker T$.
\begin{enumerate}
\item Find $a$ and compute $T(x,y,z)$ for any $(x,y,z) \in \mathbb{R}^3$.
\item Find the matrix representing $T$ with respect to standard basis.
\item Find basis for $\image T$ and $\ker T$.
\end{enumerate}

\subsubsection{Answer 8}
\label{sec-1-4-1}

\subsubsection{Answer 9}
\label{sec-1-4-2}

\subsubsection{Answer 10}
\label{sec-1-4-3}

\subsection{Problem 5}
\label{sec-1-5}
Let $T:\mathbb{R}^2 \to \mathbb{R}^2$ be a linear transformation given by
$T(x,y) = (x+2y,y)$
\begin{enumerate}
\item Find the basis $B$ of $\mathbb{R}^2$ s.t.
\begin{align*}
  [T]_B = \begin{bmatrix}
    1 & 0 \\
    2 & 1
  \end{bmatrix} \;.
\end{align*}

\item Prove that 
\begin{align*}
  \begin{bmatrix}
    1 & 0 \\
    2 & 1
  \end{bmatrix} \sim
  \begin{bmatrix}
    1 & 2 \\
    0 & 1
  \end{bmatrix}
\end{align*}
\end{enumerate}

\subsubsection{Answer 11}
\label{sec-1-5-1}

\subsubsection{Answer 12}
\label{sec-1-5-2}

\subsection{Problem 6}
\label{sec-1-6}
Let $a, b, c \in \mathbb{R}$, prove that $A \sim B \sim C$.
\begin{align*}
  A = \begin{bmatrix}
    b & c & a \\
    c & a & b \\
    a & b & c
  \end{bmatrix},
  B = \begin{bmatrix}
    c & a & b \\
    a & b & c \\
    b & c & a
  \end{bmatrix},
  C = \begin{bmatrix}
    a & b & c \\
    b & c & a \\
    c & a & b
  \end{bmatrix}\;.
\end{align*}

\subsubsection{Answer 13}
\label{sec-1-6-1}
% Emacs 25.1.50.2 (Org mode 8.2.10)
\end{document}