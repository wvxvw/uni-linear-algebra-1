% Created 2014-10-31 Fri 21:42
\documentclass[11pt]{article}
\usepackage[utf8]{inputenc}
\usepackage[T1]{fontenc}
\usepackage{fixltx2e}
\usepackage{graphicx}
\usepackage{longtable}
\usepackage{float}
\usepackage{wrapfig}
\usepackage{rotating}
\usepackage[normalem]{ulem}
\usepackage{amsmath}
\usepackage{textcomp}
\usepackage{marvosym}
\usepackage{wasysym}
\usepackage{amssymb}
\usepackage{hyperref}
\tolerance=1000
\usepackage[utf8]{inputenc}
\usepackage[usenames,dvipsnames]{color}
\usepackage{a4wide}
\usepackage[backend=bibtex, style=numeric]{biblatex}
\usepackage{commath}
\usepackage{tikz}
\usepackage{amsmath}
\usetikzlibrary{shapes,backgrounds}
\usepackage{marginnote}
\usepackage{enumerate}
\usepackage{listings}
\usepackage{color}
\hypersetup{urlcolor=blue}
\hypersetup{colorlinks,urlcolor=blue}
\addbibresource{bibliography.bib}
\setlength{\parskip}{16pt plus 2pt minus 2pt}
\definecolor{codebg}{rgb}{0.96,0.99,0.8}
\author{Oleg Sivokon}
\date{\textit{<2014-10-31 Fri>}}
\title{Assignment 12, Linear Algebra 1}
\hypersetup{
  pdfkeywords={Assignment, Linear Algebra},
  pdfsubject={First asssignment in the course Linear Algebra 1},
  pdfcreator={Emacs 25.0.50.1 (Org mode 8.2.2)}}
\begin{document}

\maketitle
\tableofcontents


\definecolor{codebg}{rgb}{0.96,0.99,0.8}
\lstnewenvironment{maxima}{%
  \lstset{backgroundcolor=\color{codebg},
    frame=single,
    framerule=0pt,
    basicstyle=\ttfamily\scriptsize,
    columns=fixed}}{}
}
\makeatletter
\newcommand{\verbatimfont}[1]{\renewcommand{\verbatim@font}{\ttfamily#1}}
\makeatother
\verbatimfont{\small}%

\clearpage

\section{Problems}
\label{sec-1}

\subsection{Problem 1}
\label{sec-1-1}

\begin{enumerate}
\item Given the system of linear equations below:

\begin{equation*}
  \left.
    \begin{alignedat}{4}
      &  x & {}+{} & 2y       & {}+{} & az & {}={} & -3 - a \\
      &  x & {}+{} & (2 - a)y & {}-{} & z  & {}={} & 1 - a \\
      & ax & {}+{} & ay       &       &    & {}={} & 6
    \end{alignedat}
  \quad \right\} \qquad
  \begin{aligned}
    a \in \mathbb{R}
  \end{aligned}
\end{equation*}

\begin{itemize}
\item What assignments to $a$ produce no solutions?
\item What assignments to $a$ produce single solution?
\item What assignments to $a$ produce infinitely many solutions?
\end{itemize}
\end{enumerate}

\subsubsection{Answer 1}
\label{sec-1-1-1}

First, express $x$, $y$ and $z$ through $a$:

\begin{maxima}
programmode: false;
linsystem: [  x + 2*y       - a*z = -3 - a,
              x + (2 - a)*y -   z = 1  - a,
            a*x + a*y             = 6];
print(linsolve(linsystem, [x, y, z]));
\end{maxima}

\begin{verbatim}
Solution:
                                 3      2
                                a  - 8 a  + 9 a - 12
                          x = - --------------------
                                     3    2
                                    a  - a  + a
                                    2
                                   a  + 3 a + 2
                               z = ------------
                                     2
                                    a  - a + 1
                                 3      2
                                a  - 2 a  + 3 a - 6
                            y = -------------------
                                     3    2
                                    a  - a  + a
[%t1, %t2, %t3]
\end{verbatim}

It's easy to see now that by solving $a^3-a^2+a=0$ we can find such an
assignment to $a$, which produces no solutions for the system.

\begin{maxima}
programmode: false;
print(solve([a^3 - a^2 + a = 0], a));
\end{maxima}

\begin{verbatim}
solve: solution:
                                   sqrt(3) %i - 1
                             a = - --------------
                                         2
                                  sqrt(3) %i + 1
                              a = --------------
                                        2
                                     a = 0
[%t1, %t2, %t3] 
\end{verbatim}

Ignoring the complex solutions, we are left with $a=0$ assignement.  Let's
verify it:

\begin{maxima}
programmode: false;
a: 0;
linsystem: [  x + 2*y       - a*z = -3 - a,
              x + (2 - a)*y -   z = 1  - a,
            a*x + a*y             = 6];
print(linsolve(linsystem, [x, y, z]));
\end{maxima}

\begin{verbatim}
[] 
\end{verbatim}

Indeed, produces no solutions.

To find out whether there might be infinite solutions, let's first find
the reduced echelon form of the system:

\begin{maxima}
programmode: false;
linsystem: [  x + 2*y       - a*z = -3 - a,
              x + (2 - a)*y -   z = 1  - a,
            a*x + a*y             = 6];
print(triangularize(coefmatrix(linsystem, [x, y, z])));
\end{maxima}

\begin{verbatim}
[ 1   2       - a     ]
[                     ]
[              2      ]
[ 0  - a      a       ] 
[                     ]
[          3    2     ]
[ 0   0   a  - a  + a ]
\end{verbatim}

By examining the reduced echelon form, we can see that infinitely many
solutions are only possible if $a$ was equal to 0, but we know that at 0
system has no solutions.  Thus any assignment to $a$ will produce a unique
solution.
\subsection{Problem 2}
\label{sec-1-2}

\begin{enumerate}
\item Given the system of linear equations below:

\begin{equation*}
  \left.
    \begin{alignedat}{5}
      &  x & {}+{} & ay        & {}+{} & bz        & {}+{} & aw         & {}={} & b \\
      &  x & {}+{} & (a + 1)y  & {}+{} & (a + b)z  & {}+{} & (a + b)w   & {}={} & a + b \\
      & ax & {}+{} & a^2y      & {}+{} & (ab + 1)z & {}+{} & (a + a^2)w & {}={} & b + ab \\
      & 2x & {}+{} & (2a + 1)y & {}+{} & (a + 2b)z & {}+{} & aw         & {}={} & 2b - 2a - ab
    \end{alignedat}
  \quad \right\} \qquad
  \begin{aligned}
    a, b \in \mathbb{R}
  \end{aligned}
\end{equation*}

\begin{itemize}
\item What assignments to $a$ and $b$ produce no solutions?
\item What assignments to $a$ and $b$ produce single solution?
\item What assignments to $a$ and $b$ produce infinitely many solutions?
\end{itemize}
\end{enumerate}

\subsubsection{Answer 2}
\label{sec-1-2-1}

First, express $a$ and $b$ through $x$, $y$ and $z$:

\begin{maxima}
programmode: false;
linsystem: [  x + a*y         + b*z         + a*w         = b,
              x + (a + 1)*y   + (a + b)*z   + (a + b)*w   = a   + b,
            a*x + a^2*y       + (a*b + 1)*z + (a + a^2)*w = b   + a*b,
            2*x + (2*a + 1)*y + (a + 2*b)*z  + a*w        = 2*b - 2*a - a*b];
print(linsolve(linsystem, [x, y, z, w]));
\end{maxima}

\begin{verbatim}
Solution:
         3         2           2     4    3      2             4    3      2
        b  + (- 3 a  + a - 1) b  + (a  - a  - 4 a  - a) b + 3 a  + a  + 3 a
  x = - --------------------------------------------------------------------
                                       b + a
                               2         2         2
                              b  + (a - a ) b - 3 a
                          z = ----------------------
                                      b + a
                           2       3    2               3    2
                      2 a b  + (- a  + a  + 2 a) b - 3 a  - a
                y = - ----------------------------------------
                                       b + a
                                     a b + 3 a
                                 w = ---------
                                       b + a
[%t1, %t2, %t3, %t4]
\end{verbatim}

One can see that assignment $a = -b$, the system has no solutions.

Since reduced echelon form of this system is:

\begin{maxima}
programmode: false;
linsystem: [  x + a*y         + b*z         + a*w         = b,
              x + (a + 1)*y   + (a + b)*z   + (a + b)*w   = a   + b,
            a*x + a^2*y       + (a*b + 1)*z + (a + a^2)*w = b   + a*b,
            2*x + (2*a + 1)*y + (a + 2*b)*z + a*w         = 2*b - 2*a - a*b];
print(triangularize(coefmatrix(linsystem, [x, y, z, w])));
\end{maxima}

\begin{verbatim}
[ 1  a  b     a    ]
[                  ]
[ 0  1  a     b    ]
[                  ] 
[ 0  0  1     a    ]
[                  ]
[ 0  0  0  - b - a ]
\end{verbatim}

In order to find an assignment, which would eliminate one pivot from reduced
echelon form, we would need to solve $-b - a = 0$, but this is exactly the
assignment which gives single solution.  So, as before, there appear to be
no assignment that produces infinitely many solutions.
\section{Exercises}
\label{sec-2}
Given $O$ is a homogeneous system of linear equations, and $M$ is not
homogeneous system of linear equations, which share the coefficients of the
row vectors of their respective matrices sans the last one.  Both $O$ and $M$
have $m$ equations and $n$ unknowns.

\begin{itemize}
\item \textbf{a} if only the first statement is correct.
\item \textbf{b} if only the second statement is correct.
\item \textbf{c} if both statements are correct.
\item \textbf{d} if neither statement is correct.
\end{itemize}

\subsection{Exercise 1}
\label{sec-2-1}

\begin{enumerate}
\item There are infinitely many solutions (to the system of linear equations
given below).
\item The homogeneous matrix created using the given system of linear equations
has infinitely many solutions.
\end{enumerate}

\begin{maxima}
programmode: false;
linsystem: [  a + 2*b -   c +   d = 2,
            2*a + 3*b - 3*c + 2*d = 3,
             -a -   b + 2*c -   d = -1,
            2*a + 4*b - 2*c + 3*d = 3,
            2*a + 2*b - 4*c + 2*d = 2];
linsolve(linsystem, [a, b, c, d]);
\end{maxima}

\begin{verbatim}
solve: dependent equations eliminated: (2 5)
Solution:
                                 a = 4 - 3 %r1
                                    d = - 1
                                  c = 1 - %r1
                                    b = %r1
\end{verbatim}

\emph{Answer:} \textbf{c}
\subsection{Exercise 2}
\label{sec-2-2}
\begin{enumerate}
\item The system given below has no solutions.
\item The system given below taken without its first equantion has no solutions.
\end{enumerate}

\begin{maxima}
programmode: false;
linsystem: [  a +   b +         d -   e = -1,
                          c +   d + 2*e = 2,
            3*a + 3*b + 2*c + 5*d + 2*e = 2,
            3*a + 3*b + 4*c + 7*d + 6*e = 2];
linsolve(linsystem, [a, b, c, d, e]);
\end{maxima}

\begin{maxima}
programmode: false;
linsystem: [              c +   d + 2*e = 2,
            3*a + 3*b + 2*c + 5*d + 2*e = 2,
            3*a + 3*b + 4*c + 7*d + 6*e = 2];
linsolve(linsystem, [a, b, c, d, e]);
\end{maxima}

\emph{Answer:} \textbf{c}
\subsection{Exercise 3}
\label{sec-2-3}

\begin{maxima}
programmode: false;
linsystem: [  x + 2*y -  3*z = a,
            2*x + 6*y - 11*z = b,
              x - 2*y +  7*z = c];
linsolve(linsystem, [a, b, c]);
\end{maxima}

\begin{verbatim}
Solution:
                              a = - 3 z + 2 y + x
                            b = - 11 z + 6 y + 2 x
                               c = 7 z - 2 y + x
\end{verbatim}


\begin{enumerate}
\item There exist such $a$, $b$ and $c$, which are the unique solution to
the system.
\item There are such $a$, $b$ and $c$, which are not a solution of the system.
\end{enumerate}

Assignment $a = 1$, $b = 1$ and $c = 1$ gives no solutions.

\begin{maxima}
programmode: false;
solution: triangularize(coefmatrix(
[ -3*x - 11*y + 7*z = a,
   2*x +  6*y - 2*z = b,
     x +  2*y +   z = c],
     [x, y, z]));
print(solution);
\end{maxima}

\begin{verbatim}
[ - 3  - 11   7  ]
[                ]
[  0    4    - 8 ] 
[                ]
[  0    0     0  ]
\end{verbatim}

Triangulated matrix of the above solution doesn't have pivot in the third
column, thus it doesn't have a unique solution.

\emph{Answer:} \textbf{b}
\subsection{Exercise 4}
\label{sec-2-4}
\begin{enumerate}
\item If $O$ has infinitely many solutions, then $n \geq m$.
\item If $n > m$, then $M$ has infinitely many solutions.
\end{enumerate}

(1) Not necessarily so because it is possible to have dependent equations.
We could simply repeat the same euqation $n+1$ times to find a counterexample.

(2) Not necessarily so because it is possible to have such matrices, which don't
have solutions at all.

\emph{Answer:} \textbf{d}
\subsection{Exercise 5}
\label{sec-2-5}
\begin{enumerate}
\item If $\vec{c}$, $\vec{d}$ are solutions of $M$, and $\mu \vec{d}$, 
      $\lambda \vec{c}$ are solutions of $M$ then $\lambda + \mu = 1$.
\item If $\vec{c}$ is a solution of $M$ and $\vec{d}$ is a solution of $O$, then
      $\vec{c} - 3\vec{d}$ is a solution of $M$.
\end{enumerate}

(1) Would be true, if $\vec{d}$ and $\vec{c}$ were the same vector and $M$ had only
one solution thus.  But if $\vec{c}$ and $\vec{d}$ are distinct, this warrants
infinitely many solutions, thus there is no requirement that a scalar multiplier
of the elementary operations performed on the solution be any particular value.

(2) If $\vec{c}$ is a unique solution of $M$, then $O$ has a unique solution too.
Since a solution of homogenous matrix is a zero vector, then adding any multiple
of it won't change the value of $\vec{c}$.  But if $M$ has infinitely many
solutions, then it is possible to see $\vec{c} - 3\vec{d}$ as bein an elementary
operation, which we can accomodate in place of at least one free unknown, which
has to be present in this case.

\emph{Answer:} \textbf{b}
% Emacs 25.0.50.1 (Org mode 8.2.2)
\end{document}