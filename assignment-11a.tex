% Created 2014-11-08 Sat 23:15
\documentclass[11pt]{article}
\usepackage[utf8]{inputenc}
\usepackage[T1]{fontenc}
\usepackage{fixltx2e}
\usepackage{graphicx}
\usepackage{longtable}
\usepackage{float}
\usepackage{wrapfig}
\usepackage{rotating}
\usepackage[normalem]{ulem}
\usepackage{amsmath}
\usepackage{textcomp}
\usepackage{marvosym}
\usepackage{wasysym}
\usepackage{amssymb}
\usepackage{hyperref}
\tolerance=1000
\usepackage[utf8]{inputenc}
\usepackage[usenames,dvipsnames]{color}
\usepackage{a4wide}
\usepackage[backend=bibtex, style=numeric]{biblatex}
\usepackage{commath}
\usepackage{tikz}
\usepackage{amsmath}
\usetikzlibrary{shapes,backgrounds}
\usepackage{marginnote}
\usepackage{enumerate}
\usepackage{listings}
\usepackage{color}
\hypersetup{urlcolor=blue}
\hypersetup{colorlinks,urlcolor=blue}
\addbibresource{bibliography.bib}
\setlength{\parskip}{16pt plus 2pt minus 2pt}
\definecolor{codebg}{rgb}{0.96,0.99,0.8}
\author{Oleg Sivokon}
\date{\textit{<2014-11-07 Fri>}}
\title{Assignment 11, Linear Algebra 1}
\hypersetup{
  pdfkeywords={Assignment, Linear Algebra},
  pdfsubject={First asssignment in the course Linear Algebra 1},
  pdfcreator={Emacs 25.0.50.1 (Org mode 8.2.2)}}
\begin{document}

\maketitle
\tableofcontents


\definecolor{codebg}{rgb}{0.96,0.99,0.8}
\lstnewenvironment{maxima}{%
  \lstset{backgroundcolor=\color{codebg},
    frame=single,
    framerule=0pt,
    basicstyle=\ttfamily\scriptsize,
    columns=fixed}}{}
}
\makeatletter
\newcommand{\verbatimfont}[1]{\renewcommand{\verbatim@font}{\ttfamily#1}}
\makeatother
\verbatimfont{\small}%
\makeatletter
\renewcommand*\env@matrix[1][*\c@MaxMatrixCols c]{%
  \hskip -\arraycolsep
  \let\@ifnextchar\new@ifnextchar
  \array{#1}}
\makeatother

 \clearpage

\section{Problems}
\label{sec-1}

\subsection{Problem 1}
\label{sec-1-1}

\begin{enumerate}
\item Given the system of linear equations below:

\begin{equation*}
  \left.
    \begin{alignedat}{4}
      &  x & {}+{} & 2y       & {}+{} & az & {}={} & -3 - a \\
      &  x & {}+{} & (2 - a)y & {}-{} & z  & {}={} & 1 - a \\
      & ax & {}+{} & ay       &       &    & {}={} & 6
    \end{alignedat}
  \quad \right\} \qquad
  \begin{aligned}
    a \in \mathbb{R}
  \end{aligned}
\end{equation*}
\end{enumerate}


\begin{enumerate}
\item What assignments to $a$ produce no solutions?
\item What assignments to $a$ produce single solution?
\item What assignments to $a$ produce infinitely many solutions?
\end{enumerate}

\subsubsection{Answer 1}
\label{sec-1-1-1}

Before answering the questions, let's reduce the matrix of coefficients of
the given system of linear equations to the echelon form.  This will be
useful when talking about its properties.

\begin{equation*}
  \begin{bmatrix}[ccc|c]
    1 & 2   & a & -3-a \\
    1 & 2-a & 1 & 1-a \\
    a & a   & 0 & 6 \\
  \end{bmatrix}
  \xrightarrow{R_2 = R_2 - R_1}
  \begin{bmatrix}[ccc|c]
    1 & 2  & a   & -3-a \\
    0 & -a & 1-a & 4 \\
    a & a  & 0   & 6 \\
  \end{bmatrix}
\end{equation*}
\begin{equation*}
  \begin{bmatrix}[ccc|c]
    1 & 2  & a   & -3-a \\
    0 & -a & 1-a & 4 \\
    a & a  & 0   & 6 \\
  \end{bmatrix}
  \xrightarrow{R_3 = R_3 - aR_1}
  \begin{bmatrix}[ccc|c]
    1 & 2  & a    & -3-a \\
    0 & -a & 1-a  & 4 \\
    0 & -a & -a^2 & 6+3a+a^2 \\
  \end{bmatrix}
\end{equation*}
\begin{equation*}
  \begin{bmatrix}[ccc|c]
    1 & 2  & a    & -3-a \\
    0 & -a & 1-a  & 4 \\
    0 & -a & -a^2 & 6+3a+a^2 \\
  \end{bmatrix}
  \xrightarrow{R_3 = R_3 - R_2}
  \begin{bmatrix}[ccc|c]
    1 & 2  & a        & -3-a \\
    0 & -a & 1-a      & 4 \\
    0 & 0  & -a^2+a-1 & 2+3a+a^2 \\
  \end{bmatrix}
\end{equation*}


\begin{enumerate}
\item From just looking at the last equation in the system, we can conclude
that $a=0$ creates an inconsistent system because that would imply
$0x+0y+0z=6$, i.e. $0=6$, which is impossible.

\item In order for the system to have a single solution the matrix of its
coefficients in its echelon form must have as many pivots as there are
unknowns.  $a$ can't influence the pivot in the first row, we already
know that $-a=0$ leads to having no solutions and by solving $-a^2+a-1$
we find that it has no real roots, but $a$ is given to be real, thus
every column has a pivot, which means that in case the matrix has
solutions it must be unique.

\item This system could have less pivots than the rank of the matrix of its
coefficients if either $-a=0 \land 1-a=0$ or $-a^2+a-1=0$, first is
clearly impossible and the second doesn't have any real roots (but it
is given that $a$ is real). So this system can never have more than
one solution.
\end{enumerate}
\subsection{Problem 2}
\label{sec-1-2}

\begin{enumerate}
\item Given the system of linear equations below:

\begin{equation*}
  \left.
    \begin{alignedat}{5}
      &  x & {}+{} & ay        & {}+{} & bz        & {}+{} & aw         & {}={} & b \\
      &  x & {}+{} & (a + 1)y  & {}+{} & (a + b)z  & {}+{} & (a + b)w   & {}={} & a + b \\
      & ax & {}+{} & a^2y      & {}+{} & (ab + 1)z & {}+{} & (a + a^2)w & {}={} & b + ab \\
      & 2x & {}+{} & (2a + 1)y & {}+{} & (a + 2b)z & {}+{} & aw         & {}={} & 2b - 2a - ab
    \end{alignedat}
  \quad \right\} \qquad
  \begin{aligned}
    a, b \in \mathbb{R}
  \end{aligned}
\end{equation*}
\end{enumerate}


\begin{enumerate}
\item What assignments to $a$ and $b$ produce no solutions?
\item What assignments to $a$ and $b$ produce single solution?
\item What assignments to $a$ and $b$ produce infinitely many solutions?
\end{enumerate}

\subsubsection{Answer 2}
\label{sec-1-2-1}
As before, let's first extract the coefficient matrix and by using
Gaussian elimination bring it to echelon form:

\begin{equation*}
  \begin{bmatrix}[cccc|c]
    1 & a    & b    & a     & b \\
    1 & a+1  & a+b  & a+b   & a+b \\
    a & a^2  & ab+1 & a+a^2 & b+ab \\
    2 & 2a+1 & a+2b & a     & 2b-2a-ab \\
  \end{bmatrix}
  \xrightarrow{R_2 = R_2 - R_1}
  \begin{bmatrix}[cccc|c]
    1 & a    & b    & a     & b \\
    0 & 1    & a    & b     & a \\
    a & a^2  & ab+1 & a+a^2 & b+ab \\
    2 & 2a+1 & a+2b & a     & 2b-2a-ab \\
  \end{bmatrix}
\end{equation*}
\begin{equation*}
  \begin{bmatrix}[cccc|c]
    1 & a    & b    & a     & b \\
    0 & 1    & a    & b     & a \\
    a & a^2  & ab+1 & a+a^2 & b+ab \\
    2 & 2a+1 & a+2b & a     & 2b-2a-ab \\
  \end{bmatrix}
  \xrightarrow{R_4 = R_4 - 2R_1}
  \begin{bmatrix}[cccc|c]
    1 & a   & b    & a     & b \\
    0 & 1   & a    & b     & a \\
    a & a^2 & ab+1 & a+a^2 & b+ab \\
    0 & 1   & a    & 0     & 2a-ab \\
  \end{bmatrix}
\end{equation*}
\begin{equation*}
  \begin{bmatrix}[cccc|c]
    1 & a   & b    & a     & b \\
    0 & 1   & a    & b     & a \\
    a & a^2 & ab+1 & a+a^2 & b+ab \\
    0 & 1   & a    & 0     & 2a-ab \\
  \end{bmatrix}
  \xrightarrow{R_3 = R_3 - aR_1}
  \begin{bmatrix}[cccc|c]
    1 & a & b & a & b \\
    0 & 1 & a & b & a \\
    0 & 0 & 1 & a & b \\
    0 & 1 & a & 0 & 2a-ab \\
  \end{bmatrix}
\end{equation*}
\begin{equation*}
  \begin{bmatrix}[cccc|c]
    1 & a & b & a & b \\
    0 & 1 & a & b & a \\
    0 & 0 & 1 & a & b \\
    0 & 1 & a & 0 & 2a-ab \\
  \end{bmatrix}
  \xrightarrow{R_4 = R_4 - R_2}
  \begin{bmatrix}[cccc|c]
    1 & a & b & a & b \\
    0 & 1 & a & b & a \\
    0 & 0 & 1 & a & b \\
    0 & 0 & 0 & -b & -3a-ab \\
  \end{bmatrix}
\end{equation*}

\begin{enumerate}
\item The only case there would be no solution to this system is when
$b=0\land -3a-ab\neq0$.  Otherwise we'd have that some real number not
equal to zero equals to zero.  Suppose now that $b=0$, then if $-3a\neq0$
the system has no solutions.  Which amounts to that whenever
$a\neq0\land b=0$ the system has no solutions.
\item In order for the system to have single solution the rank of the
coefficient matrix needs to be equal to the number of unknowns of the
system.  The only way for this system to not have that property is if $b$
is zero and $-3a-ab=0$. As discussed above, if the second condition doesn't
hold, the system has no solutions, so we are only interested in all which
remains, i.e. the cases when $b\neq0$.
\item Conversely, if $b=0\land-3a-ab=0$ then we have a free variable in this 
system, and hence infinite solutions.
\end{enumerate}
\subsection{Problem 3}
\label{sec-1-3}

Solve the system of linear equations:

\begin{equation*}
  \left.
    \begin{alignedat}{5}
      &  \frac{1}{x} & {}+{} & \frac{2}{y} & {}-{} & \frac{4}{z}  & {}={} & 1 \\
      &  \frac{2}{x} & {}+{} & \frac{3}{y} & {}+{} & \frac{8}{z}  & {}={} & 0 \\
      &  \frac{1}{x} & {}+{} & \frac{9}{y} & {}-{} & \frac{10}{z} & {}={} & 5
    \end{alignedat}
    \quad \right\} \qquad
  \begin{aligned}
    x, y, z \in \mathbb{R}
  \end{aligned}
\end{equation*}

\subsubsection{Answer 3}
\label{sec-1-3-1}

Because writing coefficient matrix as reciprocals to the system unknowns will make
this unwieldy, we'll perform Gaussian elimination directly on the equations given.

\begin{equation*}
  \left.
    \begin{alignedat}{5}
      &  \frac{1}{x} & {}+{} & \frac{2}{y} & {}-{} & \frac{4}{z}  & {}={} & 1 \\
      &  \frac{2}{x} & {}+{} & \frac{3}{y} & {}+{} & \frac{8}{z}  & {}={} & 0 \\
      &  \frac{1}{x} & {}+{} & \frac{9}{y} & {}-{} & \frac{10}{z} & {}={} & 5
    \end{alignedat}
    \quad \right\}
  \xrightarrow{R_2 = R_2 - 2R_1}
  \left.
    \begin{alignedat}{5}
      &  \frac{1}{x} & {}+{} & \frac{2}{y}  & {}-{} & \frac{4}{z}  & {}={} & 1 \\
      &  0           & {}+{} & -\frac{1}{y} & {}+{} & \frac{16}{z} & {}={} & -2 \\
      &  \frac{1}{x} & {}+{} & \frac{9}{y}  & {}-{} & \frac{10}{z} & {}={} & 5
    \end{alignedat}
    \quad \right\}
\end{equation*}
\begin{equation*}
  \left.
    \begin{alignedat}{5}
      &  \frac{1}{x} & {}+{} & \frac{2}{y} & {}-{} & \frac{4}{z}  & {}={} & 1 \\
      &  0           & {}-{} & \frac{1}{y} & {}+{} & \frac{16}{z} & {}={} & -2 \\
      &  \frac{1}{x} & {}+{} & \frac{9}{y} & {}-{} & \frac{10}{z} & {}={} & 5
    \end{alignedat}
    \quad \right\}
  \xrightarrow{R_3 = R_3 + R_1}
  \left.
    \begin{alignedat}{5}
      &  \frac{1}{x} & {}+{} & \frac{2}{y}  & {}-{} & \frac{4}{z}  & {}={} & 1 \\
      &  0           & {}-{} & \frac{1}{y}  & {}+{} & \frac{16}{z} & {}={} & -2 \\
      &  0           & {}+{} & \frac{11}{y} & {}-{} & \frac{6}{z}  & {}={} & 6
    \end{alignedat}
    \quad \right\}
\end{equation*}
\begin{equation*}
  \left.
    \begin{alignedat}{5}
      &  \frac{1}{x} & {}+{} & \frac{2}{y}  & {}-{} & \frac{4}{z}  & {}={} & 1 \\
      &  0           & {}-{} & \frac{1}{y}  & {}+{} & \frac{16}{z} & {}={} & -2 \\
      &  0           & {}+{} & \frac{11}{y} & {}-{} & \frac{6}{z}  & {}={} & 6
    \end{alignedat}
    \quad \right\}
  \xrightarrow{R_3 = R_3 + 11R_2}
  \left.
    \begin{alignedat}{5}
      &  \frac{1}{x} & {}+{} & \frac{2}{y} & {}-{} & \frac{4}{z}   & {}={} & 1 \\
      &  0           & {}-{} & \frac{1}{y} & {}+{} & \frac{16}{z}  & {}={} & -2 \\
      &  0           & {}+{} & 0           & {}+{} & \frac{102}{z} & {}={} & -16
    \end{alignedat}
    \quad \right\}
\end{equation*}
\begin{equation*}
  \left.
    \begin{alignedat}{5}
      &  \frac{1}{x} & {}+{} & \frac{2}{y} & {}-{} & \frac{4}{z}   & {}={} & 1 \\
      &  0           & {}-{} & \frac{1}{y} & {}+{} & \frac{16}{z}  & {}={} & -2 \\
      &  0           & {}+{} & 0           & {}+{} & \frac{102}{z} & {}={} & -16
    \end{alignedat}
    \quad \right\}
  \xrightarrow{R_1 = R_1 + 2R_2}
  \left.
    \begin{alignedat}{5}
      &  \frac{1}{x} & {}+{} & 0           & {}+{} & \frac{28}{z}  & {}={} & -3 \\
      &  0           & {}-{} & \frac{1}{y} & {}+{} & \frac{16}{z}  & {}={} & -2 \\
      &  0           & {}+{} & 0           & {}+{} & \frac{102}{z} & {}={} & -16
    \end{alignedat}
    \quad \right\}
\end{equation*}
\begin{equation*}
  \left.
    \begin{alignedat}{5}
      &  \frac{1}{x} & {}+{} & 0           & {}+{} & \frac{28}{z}  & {}={} & -3 \\
      &  0           & {}-{} & \frac{1}{y} & {}+{} & \frac{16}{z}  & {}={} & -2 \\
      &  0           & {}+{} & 0           & {}+{} & \frac{102}{z} & {}={} & -16
    \end{alignedat}
    \quad \right\}
  \xrightarrow{R_2 = R_2 - \frac{14}{51}R_3}
  \left.
    \begin{alignedat}{5}
      &  \frac{1}{x} & {}+{} & 0           & {}+{} & \frac{28}{z}  & {}={} & -3 \\
      &  0           & {}-{} & \frac{1}{y} & {}+{} & 0  & {}={}    & \frac{26}{51} \\
      &  0           & {}+{} & 0           & {}+{} & \frac{102}{z} & {}={} & -16
    \end{alignedat}
    \quad \right\}
\end{equation*}
\begin{equation*}
  \left.
    \begin{alignedat}{5}
      &  \frac{1}{x} & {}+{} & 0           & {}+{} & \frac{28}{z}  & {}={} & -3 \\
      &  0           & {}-{} & \frac{1}{y} & {}+{} & 0             & {}={} & \frac{26}{51} \\
      &  0           & {}+{} & 0           & {}+{} & \frac{102}{z} & {}={} & -16
    \end{alignedat}
    \quad \right\}
  \xrightarrow{R_1 = R_1 - \frac{8}{51}R_3}
  \left.
    \begin{alignedat}{5}
      &  \frac{1}{x} & {}+{} & 0           & {}+{} & 0             & {}={} & \frac{71}{51} \\
      &  0           & {}-{} & \frac{1}{y} & {}+{} & 0             & {}={} & \frac{26}{51} \\
      &  0           & {}+{} & 0           & {}+{} & \frac{102}{z} & {}={} & -16
    \end{alignedat}
    \quad \right\}
\end{equation*}

Now we can extract the variables:

\begin{equation*}
  x = \frac{51}{71}, \quad
  y = -\frac{51}{26}, \quad
  z = -\frac{102}{16}
\end{equation*}

Let's verify:

\begin{equation*}
  \begin{split}
    \frac{1}{x} + \frac{2}{y} - \frac{4}{z} = 1 \\
    \frac{71}{51} - \frac{52}{51} + \frac{64}{102} = 1 \\
    \frac{19}{51} - \frac{32}{51} = 1 \\
    \frac{51}{51} = 1
  \end{split}
\end{equation*}

Similarly for other cases.
\subsection{Problem 4}
\label{sec-1-4}
Given $U = \{\vec{u_1}, \vec{u_2}, \vec{u_3}, \vec{u_4}\}$ is a linearly
independant set of vectors in $\mathbb{R}^5$ and vectors:

\begin{equation*}
  \begin{alignedat}{4}
    & v_1 & {}={} & 8au_1 {}+{} & 2u_2 {}+{}   & u_3 \\
    & v_2 & {}={} &             & 16au_2 {}+{}                        & u_4 \\
    & v_3 & {}={} & u_1 {}-{}                  & \frac{1}{2}u_3 {}+{} & au_4 \\
    & a \in \mathbb{R}
  \end{alignedat}
\end{equation*}

\begin{enumerate}
\item Find all $a$ such that $V = \{v_1, v_2, v_3\}$ is linearly dependent.
\item For every $a$ found in (1), write $v_2$ as linear combination of $v_1$
and $v_3$.
\item Is it possible to adjoin the vectors $v_i$ to $U$ such that 
      $U \cup \{v_i\}$ would become a basis in $\mathbb{R}$?
\end{enumerate}

\subsubsection{Answer 4}
\label{sec-1-4-1}

First we will arrange all coefficients describing vectors $v_i$ as rows of
the matrix.  Since in order to find a linearly dependent combination of rows
we need the matrix to be homogenous, the last row of the matrix is the zero
vector.  Thus, I'll only write the "interesting" columns.  I will reduce
this matrix to the echelon form in order to find possible contradictions
(possible contradictions are rows containing single coefficient).  These
raws will yield equations, which, if solved, will give values of $a$ required
for the system to have solutions.  This will be equivalent to finding values
of $a$ s.t. they make linear combination of vectors $v_i$ linearly dependant.

\begin{equation*}
  \begin{bmatrix}[ccc]
    8a & 0   & 1 \\
    2  & 16a & 0 \\
    1  & 0   & -\frac{1}{2} \\
    0  & 1   & a \\
  \end{bmatrix}
  \xrightarrow{R_1 = R_2, R_2 = R_1}
  \begin{bmatrix}[ccc]
    2  & 16a & 0 \\
    8a & 0   & 1 \\
    1  & 0   & -\frac{1}{2} \\
    0  & 1   & a \\
  \end{bmatrix}
  \xrightarrow{R_2 = R_2 + 4aR_1}
  \begin{bmatrix}[ccc]
    2 & 16a   & 0 \\
    0 & 64a^2 & 1 \\
    1 & 0     & -\frac{1}{2} \\
    0 & 1     & a \\
  \end{bmatrix}
\end{equation*}
\begin{equation*}
  \begin{bmatrix}[ccc]
    2 & 16a   & 0 \\
    0 & 64a^2 & 1 \\
    1 & 0     & -\frac{1}{2} \\
    0 & 1     & a \\
  \end{bmatrix}
  \xrightarrow{R_3 = 2R_3}
  \begin{bmatrix}[ccc]
    2 & 16a   & 0 \\
    0 & 64a^2 & 1 \\
    2 & 0     & -1 \\
    0 & 1     & a \\
  \end{bmatrix}
  \xrightarrow{R_3 = R_3 - R_1}
  \begin{bmatrix}[ccc]
    2 & 16a   & 0 \\
    0 & 64a^2 & 1 \\
    0 & -16a  & -1 \\
    0 & 1     & a \\
  \end{bmatrix}
\end{equation*}
\begin{equation*}
  \begin{bmatrix}[ccc]
    2 & 16a   & 0 \\
    0 & 64a^2 & 1 \\
    0 & -16a  & -1 \\
    0 & 1     & a \\
  \end{bmatrix}
  \xrightarrow{R_3 = R_2, R_2 = R_3}
  \begin{bmatrix}[ccc]
    2 & 16a   & 0 \\
    0 & -16a  & -1 \\
    0 & 64a^2 & 1 \\
    0 & 1     & a \\
  \end{bmatrix}
  \xrightarrow{R_3 = R_3 - 4aR_2}
  \begin{bmatrix}[ccc]
    2 & 16a  & 0 \\
    0 & -16a & -1 \\
    0 & 0    & 1-4a \\
    0 & 1    & a \\
  \end{bmatrix}
\end{equation*}
\begin{equation*}
  \begin{bmatrix}[ccc]
    2 & 16a  & 0 \\
    0 & -16a & -1 \\
    0 & 0    & 1-4a \\
    0 & 1    & a \\
  \end{bmatrix}
  \xrightarrow{R_4 = 16R_4}
  \begin{bmatrix}[ccc]
    2 & 16a  & 0 \\
    0 & -16a & -1 \\
    0 & 0    & 1-4a \\
    0 & 16a  & 16a^2 \\
  \end{bmatrix}
  \xrightarrow{R_4 = R_4 - R_2}
  \begin{bmatrix}[ccc]
    2 & 16a  & 0 \\
    0 & -16a & -1 \\
    0 & 0    & 1-4a \\
    0 & 0    & 16a^2-1 \\
  \end{bmatrix}
\end{equation*}

Which gives us two candidate equations: $1-4a=0$ and $16a^2-1=0$ with respective roots
${{1}\over{4}}$ and $-{{1}\over{4}}$.

Now we can write $v_2$ as linear combination of $v_1$ and $v_3$ for ${{1}\over{4}}$:

\begin{equation*}
\begin{alignedat}{5}
 & (0, \frac{32}{4}, 0, 0) &{}={}& x(2, 0, 0, 0)  &{}+{}& y(-1, 2, \frac{32}{4^2} - 2) \\
 & (0, 8, 0, 0)            &{}={}& x(2, 0, 0, 0)  &{}+{}& y(-1, 2, 0, 0) \\
 & (0, 8, 0, 0)            &{}={}& 4(-1, 2, 0, 0) &{}+{}& 2(2, 0, 0, 0)\\
 & v_2                     &{}={}& 4v_3           &{}+{}& 2v_1
\end{alignedat}
\end{equation*}

and similarly for $-{{1}\over{4}}$:

\begin{equation*}
\begin{alignedat}{5}
 & (0, -\frac{32}{4}, 0, 0) &{}={}& x(2, 0, 0, 0)   &{}+{}& y(-1, 2, \frac{32}{-4^2} - 2) \\
 & (0, -8, 0, 0)            &{}={}& x(2, 0, 0, 0)   &{}+{}& y(-1, 2, 0, 0) \\
 & (0, -8, 0, 0)            &{}={}& -4(-1, 2, 0, 0) &{}+{}& -2(2, 0, 0, 0)\\
 & v_2                      &{}={}& -4v_3           &{}+{}& -2v_1
\end{alignedat}
\end{equation*}

No, it is not possible to create a basis from $u_i \cup v_i$ because none of $v_i$ affects
the fifth dimension of $\mathbb{R}^5$ and because everyone of $v_i$ is a linear combination
of $u_i$, none of $u_i$ could have any effect on the fifth dimension either.

\subsection{Problem 5}
\label{sec-1-5}
Given $\vec{a_1}, ..., \vec{a_k}$ and $\vec{b}$ all in $\mathbb{R}^n$.  Also
given that $\vec{b} \neq 0$ and all $\vec{a_1}, ..., \vec{a_k}$ are distinct.
Assume also that the equation $x_1\vec{a}_1+...+x_k\vec{a}_k=\vec{b}$ has
infinitely many solutions.

Prove or disprove:

\begin{enumerate}
\item If $k \geq n+1$, then $\{\vec{a_1}, ..., \vec{a_k}\}$ spans $\mathbb{R}^n$.
\item $\{\vec{a_1}, ..., \vec{a_k}\}$ is linearly dependant.
\item Exists $\vec{c} \in \mathbb{R}^n$ s.t. 
      $x_1\vec{a}_1+...+x_k\vec{a}_k=\vec{c}$ has unique solution.
\end{enumerate}

\subsubsection{Answer 5}
\label{sec-1-5-1}
(1) $\{\vec{a_1}, ..., \vec{a_k}\}$ doesn't necessary span $\mathbb{R}^n$.
In order to span a field of a dimension $n$, this set has to have at least
$n$ pivot elements in its coefficient matrix.  This can only happen when
there are at least $n$ linearly independant vectors (but we are only given
that they are distinct, not necessarily independant).  More so, we are given
that there exists $\vec{b}$, which guarantees that the rank of the matrix of
the coefficients will be at least one point short (but possibly more) of
representing a spanning set.

To convince yourself this is actually possible, let's construct such vectors
for $k=3$, $n=2$.

\begin{equation*}
  \begin{split}
    \vec{a_1} = (0, 0)\\
    \vec{a_2} = (0, 1)\\
    \vec{a_3} = (0, 2)\\
    \vec{b} = (0, 3)\\
  \end{split}
\end{equation*}


The matrix of coefficients of this set of vectors would be:

\begin{equation*}
  \begin{bmatrix}[ccc|c]
    0 & 0 & 0  & 0 \\
    0 & 1 & 2  & 3 \\
  \end{bmatrix}
\end{equation*}


This matrix has infinitely many solutions (because it lacks a pivot in the
first column), so it satisfies the requirement, but the vectors used to
construct its columns are clearly not a spanning set of $\mathbb{R}^2$
(becuase the first element of $\mathbb{R}^2$ is never assigned to).

(2) Yes, $\{\vec{a_1}, ..., \vec{a_k}\}$ is linearly dependant.  This is
warranted by infinite number of solutions to the equation describing the sum
of the vectors, by Rouché–Capelli theorem.

(3) No, there can't be a $\vec{c}$ that would force this system to have a
unique solution.  The number of solutions of a system is the property of its
augmented matrix and its coefficient matrix, neither of which include
$\vec{c}$.
% Emacs 25.0.50.1 (Org mode 8.2.2)
\end{document}