% Created 2016-03-19 Sat 12:06
\documentclass[11pt]{article}
\usepackage[utf8]{inputenc}
\usepackage[T1]{fontenc}
\usepackage{fixltx2e}
\usepackage{graphicx}
\usepackage{longtable}
\usepackage{float}
\usepackage{wrapfig}
\usepackage{rotating}
\usepackage[normalem]{ulem}
\usepackage{amsmath}
\usepackage{textcomp}
\usepackage{marvosym}
\usepackage{wasysym}
\usepackage{amssymb}
\usepackage{capt-of}
\usepackage[hidelinks]{hyperref}
\tolerance=1000
\usepackage[utf8]{inputenc}
\usepackage[usenames,dvipsnames]{color}
\usepackage{a4wide}
\usepackage{commath}
\usepackage{amsmath}
\usepackage{marginnote}
\usepackage{enumerate}
\usepackage{listings}
\usepackage{color}
\hypersetup{urlcolor=blue}
\hypersetup{colorlinks,urlcolor=blue}
\setlength{\parskip}{16pt plus 2pt minus 2pt}
\definecolor{codebg}{rgb}{0.96,0.99,0.8}
\author{Oleg Sivokon}
\date{\textit{<2016-02-27 Sat>}}
\title{Assignment 11, Linear Algebra 1}
\hypersetup{
 pdfauthor={Oleg Sivokon},
 pdftitle={Assignment 11, Linear Algebra 1},
 pdfkeywords={Assignment, Linear Algebra},
 pdfsubject={First asssignment in the course Linear Algebra 1},
 pdfcreator={Emacs 25.0.50.1 (Org mode 8.3beta)}, 
 pdflang={English}}
\begin{document}

\maketitle
\tableofcontents

\definecolor{codebg}{rgb}{0.96,0.99,0.8}
\lstnewenvironment{maxima}{%
  \lstset{backgroundcolor=\color{codebg},
    frame=single,
    framerule=0pt,
    basicstyle=\ttfamily\scriptsize,
    columns=fixed}}{}
}
\makeatletter
\newcommand{\verbatimfont}[1]{\renewcommand{\verbatim@font}{\ttfamily#1}}
\makeatother
\verbatimfont{\small}%
\makeatletter
\renewcommand*\env@matrix[1][*\c@MaxMatrixCols c]{%
  \hskip -\arraycolsep
  \let\@ifnextchar\new@ifnextchar
  \array{#1}}
\makeatother
\clearpage

\section{Problems}
\label{sec:orgheadline19}

\subsection{Problem 1}
\label{sec:orgheadline3}
Solve given systems of linear equations:

\begin{equation*}
  \left.
    \begin{alignedat}{4}
      & 2x & {}-{} & y  & {}+{} & 4z & {}={} & 1 \\
      &  x & {}+{} & 2y & {}-{} & 3z & {}={} & 6 \\
      &  x & {}-{} & y  & {}+{} &  z & {}={} & 3
    \end{alignedat}
    \quad \right\} \qquad
  \begin{aligned}
    x,y,z \in \mathbb{R}
  \end{aligned}
\end{equation*}

\begin{equation*}
  \left.
    \begin{alignedat}{5}
      & 2x & {}+{} & 2y & {}+{} & 8z & {}+{} & w   & {}={} & 0 \\
      & 3x & {}+{} & 3y & {}+{} & 3z & {}+{} & 13w & {}={} & 0 \\
      & 2x & {}+{} & 2y & {}+{} & 4z & {}+{} & 3w  & {}={} & 0
    \end{alignedat}
    \quad \right\} \qquad
  \begin{aligned}
    x,y,z,w \in \mathbb{R}
  \end{aligned}
\end{equation*}

Which variables are bound, which are free?

\subsubsection{Answer 1}
\label{sec:orgheadline1}

\begin{verbatim}
programmode: false;
linsystem: [  2*x -   y + 4*z = 1,
                x + 2*y - 3*z = 6,
                x -   y +   z = 3];
print(linsolve(linsystem, [x, y, z]));
\end{verbatim}

\begin{verbatim}
Solution:
                                        17
                                    x = --
                                        5
                                          7
                                    y = - -
                                          5
                                          9
                                    z = - -
                                          5
[%t1, %t2, %t3]
\end{verbatim}

First, we subtract third equation from the second in order to express \(y\) in
terms of \(z\):

\begin{align*}
(x + 2y - 3z) - (x - y + z) &= 6 - 3 \iff \\
                    3y - 4z &= 3 \iff \\
                          y &= \frac{3+4z}{3}\;.
\end{align*}

Next, we do the same for third and first equation to express \(x\) in terms of
\(z\):

\begin{align*}
(2x - y + 4z) - (x - y + z) &= 1 - 3 \iff \\
                    x + 3z &= 2 \iff \\
                          x &= -2 - 3z\;.
\end{align*}

Then we substitute \(x\) and \(y\) into the third equation:

\begin{align*}
-2 - 3z - \frac{3 + 4z}{3} + z &= 3 \iff \\
        -2z - 1 - \frac{4}{3}z &= 5 \iff \\
                -\frac{10}{3}z &= 6 \iff \\
                             z &= -\frac{9}{5}\;.
\end{align*}

Similarly, we find \(x = \frac{17}{5}\) and \(y = -\frac{7}{5}\).

Substituting the results into original system gives:

\begin{align*}
  \left.
    \begin{alignedat}{4}
      & 2\frac{17}{5} & {}-{} & -\frac{7}{5}  & {}+{} & -4\frac{9}{5} & {}={} & 1 \\
      &  \frac{17}{5} & {}+{} & -2\frac{7}{5} & {}-{} & -3\frac{9}{5} & {}={} & 6 \\
      &  \frac{17}{5} & {}-{} & -\frac{7}{5}  & {}+{} & -\frac{9}{5}  & {}={} & 3
    \end{alignedat}
    \quad \right\} &
  \iff \\
  \left.
  \begin{alignedat}{4}
    & 34 & {}+{} & 7  & {}-{} & 36 & {}={} & 5 \\
    & 17 & {}-{} & 14 & {}+{} & 27 & {}={} & 30 \\
    & 17 & {}+{} & 7  & {}-{} &  9 & {}={} & 15
  \end{alignedat}
  \quad \right\} & 
  \iff \\
  \left.
  \begin{alignedat}{4}
    & 5 & {}={} & 5 \\
    & 30 & {}={} & 30 \\
    & 15 & {}={} & 15
  \end{alignedat}
  \quad \right\} & \;.
\end{align*}

If we convert the given system to a matrix and bring the matrix to the
row-echelon form we get:

\begin{align*}
  \begin{bmatrix}
    2 & -1 & 4 \\
    1 & 2  & -3 \\
    1 & -1 & 1
  \end{bmatrix}
  \begin{aligned} \xrightarrow{R_3 = R_3 - R_2} \end{aligned}
  \begin{bmatrix}
    2 & -1 & 4 \\
    1 & 2  & -3 \\
    0 & -3 & 4
  \end{bmatrix}
  \begin{aligned} \xrightarrow{R_1 = R_1 - 2R_2} \end{aligned} \\
  \begin{bmatrix}
    0 & -3 & 10 \\
    1 & 2  & -3 \\
    0 & -3 & 4
  \end{bmatrix}
  \begin{aligned} \xrightarrow{R_1 = R_2, R_2 = R_1} \end{aligned}
  \begin{bmatrix}
    1 & 2  & -3 \\
    0 & -3 & 10 \\
    0 & -3 & 4
  \end{bmatrix}
  \begin{aligned} \xrightarrow{R_3 = R_3 - R_2} \end{aligned}
  \begin{bmatrix}
    1 & 2  & -3 \\
    0 & -3 & 10 \\
    0 & 0 & -6
  \end{bmatrix}
\end{align*}

We can see that all columns have leading variables, thus there are no free
variables.

\subsubsection{Answer 2}
\label{sec:orgheadline2}

\begin{verbatim}
programmode: false;
linsystem: [  2*x + 2*y + 8*z +    w = 0,
              3*x + 3*y + 3*z + 13*w = 0,
              2*x + 2*y + 4*z +  3*w = 0];
print(linsolve(linsystem, [x, y, z, w]));
\end{verbatim}

\begin{verbatim}
Solution:
                                   x = - %r1
                                     z = 0
                                     w = 0
                                    y = %r1
[%t1, %t2, %t3, %t4] 
\end{verbatim}

Similarly to the \ref{sec:orgheadline1}, we first express \(w\) in terms of \(z\):

\begin{align*}
2x + 2y + 8z + w - 2x - 2y - 4z - 3w &= 0 \iff \\
4z - 2w &= 0 \iff \\
w = &= 2z\;.
\end{align*}

Now we can rewrite the system as:

\begin{equation*}
  \left.
    \begin{alignedat}{4}
      & 2x & {}+{} & 2y & {}+{} & 10z & {}={} & 0 \\
      & 3x & {}+{} & 3y & {}+{} & 29z & {}={} & 0 \\
      & 2x & {}+{} & 2y & {}+{} & 10z & {}={} & 0
    \end{alignedat}
    \quad \right\} \qquad
  \begin{aligned}
    x,y,z \in \mathbb{R}
  \end{aligned}
\end{equation*}

which is essentially the same as:

\begin{equation*}
  \left.
    \begin{alignedat}{4}
      & 2x & {}+{} & 2y & {}+{} & 10z & {}={} & 0 \\
      & 3x & {}+{} & 3y & {}+{} & 29z & {}={} & 0 
    \end{alignedat}
    \quad \right\} \qquad
  \begin{aligned}
    x,y,z \in \mathbb{R}
  \end{aligned}
\end{equation*}

Expressing \(x\) in terms of \(y\) and \(z\) gives:

\begin{align*}
  3x + 3y + 29z - 2x - 2y - 10z &= 0 \iff \\
  x + y + 19z &= 0 \iff \\
  x &= -y - 19z\;.
\end{align*}

Substituting it back into firxt equation to solve for \(y\):

\begin{align*}
  3(-y - 19z) + 3y + 29z &= 0 \iff \\
  3y - 3y - 57z + 29z &= 0 \iff \\
  28z &= 0 \iff \\
  z &= 0\;.
\end{align*}

Now we substitute this result back into our description of \(x\), thus
obtaining:

\begin{align*}
x &= -y - 19 \times 0 \iff \\
x &= -y\;.
\end{align*}

Which is the solution for the given system of linear equations.

We'll bring the matrix corresponding to this system to the row-echelon form
to find the free and the bound variables.

\begin{align*}
  \begin{bmatrix}
    2 & 2 & 8 & 1 \\
    3 & 3 & 3 & 13 \\
    2 & 2 & 4 & 3
  \end{bmatrix}
  \begin{aligned} \xrightarrow{R_3 = R_3 - R_1} \end{aligned}
  \begin{bmatrix}
    2 & 2 & 8 & 1 \\
    3 & 3 & 3 & 13 \\
    0 & 0 & -4 & 2
  \end{bmatrix}
  \begin{aligned} \xrightarrow{R_2 = 2R_2} \end{aligned} \\
  \begin{bmatrix}
    2 & 2 & 8 & 1 \\
    6 & 6 & 6 & 26 \\
    0 & 0 & -4 & 2
  \end{bmatrix}
  \begin{aligned} \xrightarrow{R_2 = R_2 - 3R_1} \end{aligned}
  \begin{bmatrix}
    2 & 2 & 8   & 1 \\
    0 & 0 & -18 & 23 \\
    0 & 0 & -4  & 2
  \end{bmatrix}
  \begin{aligned} \xrightarrow{R_3 = 9R_3} \end{aligned} \\
  \begin{bmatrix}
    2 & 2 & 8   & 1 \\
    0 & 0 & -18 & 23 \\
    0 & 0 & -36  & 18
  \end{bmatrix}
  \begin{aligned} \xrightarrow{R_3 = R_3 - 2R_2} \end{aligned}
  \begin{bmatrix}
    2 & 2 & 8   & 1 \\
    0 & 0 & -18 & 23 \\
    0 & 0 & 0   & -28
  \end{bmatrix}
\end{align*}

Since the second column doesn't have a pivot element, I conclude that \(y\) is
free in this linear system, while the rest of the variables are bound.

\subsection{Problem 2}
\label{sec:orgheadline7}

For the given system:

\begin{equation*}
  \left.
    \begin{alignedat}{4}
      & x  & {}+{} & ay   & {}+{} & z & {}={} & 1 \\
      & ax & {}+{} & a^2y & {}+{} & z & {}={} & 2+a \\
      & ax & {}+{} & 3ay  & {}+{} & z & {}={} & 2-t
    \end{alignedat}
    \quad \right\} \qquad
  \begin{aligned}
    a,t,x,y,z \in \mathbb{R}
  \end{aligned}
\end{equation*}

\begin{enumerate}
\item Find \(a, t\) s.t. the system has a unique solution.
\item Find \(a, t\) s.t. the system has infinitely many solutions.
\item Find \(a, t\) s.t. the system has no solutions.
\end{enumerate}

\subsubsection{Answer 3}
\label{sec:orgheadline4}
We could first reduce the matrix representing this system to the row-echelon
form:

\begin{align*}
  \begin{bmatrix}
    1 & a   & 1 \\
    a & a^2 & 1 \\
    a & 3a  & a
  \end{bmatrix}
  \begin{aligned} \xrightarrow{R_3 = R_3 - R_2} \end{aligned}
  \begin{bmatrix}
    1 & a      & 1 \\
    a & a^2    & 1 \\
    0 & 3a-a^2 & a
  \end{bmatrix}
  \begin{aligned} \xrightarrow{R_2 = R_2 - aR_2} \end{aligned}
  \begin{bmatrix}
    1 & a      & 1 \\
    0 & 0      & 1-a \\
    0 & 3a-a^2 & a
  \end{bmatrix}
\end{align*}

From which we conclude that whenever \(1-a \neq 0\) and \(3a-a^2 \neq 0\) there
would be a pivot element in every column, thus ensuring the system has
exactly one solution.

Second equation factors as \(a(3-a)\), thus its roots are \(a=0\) and \(a=3\).
Subsequently, whenever \(a \neq 1\) and \(a \neq 0\) and \(a \neq 3\) the system
has a unique solution.

\subsubsection{Answer 4}
\label{sec:orgheadline5}
If we put \(a = 3, t = -3\) then the system has infinitely many solutions since
the second and the third its equations become multiples of each other:

\begin{align*}
  3x + 3^2y + z &= 2+3 \\
    &\textit{while, at the same time} \\
  3x + 3 \times 3 + z &= 2-(-3) \\
    &\textit{simplifying both parts} \\
  3x + 9y + z &= 5 \\
    &\textit{and} \\
  3x + 9y + z &= 5\;.
\end{align*}

\subsubsection{Answer 5}
\label{sec:orgheadline6}
It is easy to see that whenever \(a = 1\), no matter the value of \(t\), the
system is inconsistent:

\begin{align*}
  x + 1 \times y + z &= 1 \\
    &\textit{while, at the same time} \\
  1 \times x + 1^2y + z &= 2+1 \\
    &\textit{subtracting both parts} \\
  x + y + z - x - y - z &= 1 - 3 \iff \\
  0 &= -2\;.
\end{align*}

Another case when the system becomes inconsisten is when \(a = 0\) and \(t \neq
    0\), since the third and the second equations would become inconsistent:

\begin{align*}
  0x + 0^2y + z &= 2+0 \\
    &\textit{while, at the same time} \\
  0x + 3 \times 0 + z &= 2-t \\
    &\textit{simplifying both parts} \\
  z &= 2 \\
    &\textit{and} \\
  z &= 2-t\;.
\end{align*}

\subsection{Problem 3}
\label{sec:orgheadline10}
Given that vectors \(\vec{v}=(4, -2, -2, 4)\) and \(\vec{u}=(-2, 4, 4, -2)\) are
solutions to the system of linear equations \(M\) with four unknowns.  Also
known that \((2, 2, 2, 2)\) isn't a solution of \(M\).

\begin{enumerate}
\item Prove that the system isn't homogeneous.
\item Prove that \((0, 2, 2, 0)\) is also a solution of the system.
\end{enumerate}

\subsubsection{Answer 6}
\label{sec:orgheadline8}
Suppose, for contradiction, \(M\) is homogeneous.  Then it must be the case
that any linear combination of \(\vec{v}\) and \(\vec{u}\) is also a solution
to the system.  In particular, \(\vec{v}+\vec{u}\) is such a solution, but
\(\vec{v}+\vec{u} = (2, 2, 2, 2)\), contrary to the given.

Hence, by contradiction, \(M\) is not homogeneous.

\subsubsection{Answer 7}
\label{sec:orgheadline9}
Since the set of all solutions to the linear system is closed under
multiplication by a scalar, it is possible that \(\vec{v}\), \(\vec{u}\), or
their linear combination multiplied by a scalar will result in \((0, 2, 2,
    0)\), and indeed, \(\frac{1}{3}(\vec{v}+2\vec{u})=(0, 2, 2, 0)\).

Hence, \((0, 2, 2, 0)\) is a solution of \(M\).

\subsection{Problem 4}
\label{sec:orgheadline14}
Let \(\{\vec{u_1}, \vec{u_2}, \vec{u_3}, \vec{u_4}\}\) be a basis in \(\mathbb{R}^4\).

\begin{align*}
  \vec{v_1} &= k\vec{u_1} - \vec{u_3} + \vec{u_4} \\
  \vec{v_2} &= \vec{u_1} + \vec{u_2} - \vec{u_4} \\
  \vec{v_3} &= 4\vec{u_2} + k\vec{u_3} - 6\vec{u_4} \\
  \textit{where}\; k \in \mathbb{R}
\end{align*}

\begin{enumerate}
\item For what values of \(k\) vectors \(\vec{v_1}\), \(\vec{v_2}\), \(\vec{v_3}\) are
linearly (in-)dependent?
\item Whenever the above vectors are linearly dependent, write \(\vec{v_3}\) as a
combination of \(\vec{v_1}\) and \(\vec{v_2}\).
\item What are the values of \(k\) for which the set \(\{\vec{u_1}, \vec{u_2},
      \vec{u_3}, \vec{v_1}, \vec{v_2}\}\) spans \(\mathbb{R}^4\)?
\end{enumerate}

\subsubsection{Answer 8}
\label{sec:orgheadline11}
Recall that matrix comprised of column vectors adjoined to the solution
vector (zero in our case) will have single solution if the vectors are
linearly independent.  Hence, represent the \(\vec{v_i}\) first in terms of
\(\vec{u_i}\), then in matrix form:

\begin{align*}
  a_1(k\vec{u_1} - \vec{u_3} + \vec{u_4}) +
  a_2(\vec{u_1} + \vec{u_2} - \vec{u_4}) +
  a_3(\vec{u_2} + k\vec{u_3} - 6\vec{u_4}) &= \vec{0} \iff \\
  (a_1k + a_2)\vec{u_1} + 
  (a_2 + 4a_3)\vec{u_2} + 
  (ka_3 - a_1)\vec{u_3} + 
  (a_1 - a_2 - 6a_3)\vec{u_4} &= \vec{0} \iff \\
  \textit{has unique solution} &
                                 \begin{bmatrix}
                                   k  & 1  & 0 \\
                                   0  & 1  & 4 \\
                                   -1 & 0  & k \\
                                   1  & -1 & -6
                                 \end{bmatrix}
\end{align*}

It is important now to see what happens when \(k=0\) since this will affect
the first pivot element:

\begin{align*}
  \begin{bmatrix}
    0  & 1  & 0 \\
    0  & 1  & 4 \\
    -1 & 0  & 0 \\
    1  & -1 & -6
  \end{bmatrix}
  \begin{aligned} \xrightarrow{R_1 = R_4, R_4 = R_1} \end{aligned}
  \begin{bmatrix}
    1  & -1 & -6 \\
    0  & 1  & 4 \\
    -1 & 0  & 0 \\
    0  & 1  & 0
  \end{bmatrix}
  \begin{aligned} \xrightarrow{R_3 = R_3 + R_1} \end{aligned}
  \begin{bmatrix}
    1  & -1 & -6 \\
    0  & 1  & 4 \\
    0  & -1 & -6 \\
    0  & 1  & 0
  \end{bmatrix}
  \begin{aligned} \xrightarrow{R_3 = R_3 + R_2} \end{aligned} \\
  \begin{bmatrix}
    1  & -1 & -6 \\
    0  & 1  & 4 \\
    0  & 0  & -2 \\
    0  & 1  & 0
  \end{bmatrix}
  \begin{aligned} \xrightarrow{R_4 = R_4 - R_2} \end{aligned}
  \begin{bmatrix}
    1  & -1 & -6 \\
    0  & 1  & 4 \\
    0  & 0  & -2 \\
    0  & 0  & -4
  \end{bmatrix}
  \begin{aligned} \xrightarrow{R_4 = R_4 + 2R_3} \end{aligned}
  \begin{bmatrix}
    1  & -1 & -6 \\
    0  & 1  & 4 \\
    0  & 0  & -2 \\
    0  & 0  & 0
  \end{bmatrix}
\end{align*}

I.e. when \(k = 0\), the system has unique solution.  Otherwise:

\begin{align*}
  \begin{bmatrix}
    k  & 1  & 0 \\
    0  & 1  & 4 \\
    -1 & 0  & k \\
    1  & -1 & -6
  \end{bmatrix}
  \begin{aligned} \xrightarrow{R_3 = R_3 + R_4} \end{aligned}
  \begin{bmatrix}
    k & 1  & 0 \\
    0 & 1  & 4 \\
    0 & -1 & k-6 \\
    1 & -1 & -6
  \end{bmatrix}
  \begin{aligned} \xrightarrow{R_4 = kR_4 - R_1} \end{aligned}
  \begin{bmatrix}
    k & 1    & 0 \\
    0 & 1    & 4 \\
    0 & -1   & k-6 \\
    0 & -k-1 & -6
  \end{bmatrix}
  \begin{aligned} \xrightarrow{R_3 = R_3 + R_2} \end{aligned} \\
  \begin{bmatrix}
    k & 1    & 0 \\
    0 & 1    & 4 \\
    0 & 0    & k-2 \\
    0 & -k-1 & -6
  \end{bmatrix}
  \begin{aligned} \xrightarrow{R_4 = R_4 + (k+1)R_2} \end{aligned}
  \begin{bmatrix}
    k & 1 & 0 \\
    0 & 1 & 4 \\
    0 & 0 & k-2 \\
    0 & 0 & 4k-2 
  \end{bmatrix}
\end{align*}

We can see that the third and fourth equations are equivalent, and the system
only has unique solution whenever \(k \neq \frac{1}{2}\) and \(k \neq 2\).

\subsubsection{Answer 9}
\label{sec:orgheadline12}
In the way similar to the \ref{sec:orgheadline11}, we can write a system of symultaneous
equations:

\begin{align*}
  \left.
  \begin{alignedat}{3}
    & kx_1 & {}+{} & x_2 & {}={} & 0 \\
    &  &           & x_2 & {}={} & 4y \\
    & -x_1 &       &     & {}={} & ky \\
    & -x_1 & {}-{} & x_2 & {}={} & 6y
  \end{alignedat} \quad \right\} \iff \\
  -x_1 &= 6y + x_2 &\iff \\
  ky   &= 6y + x_2 &\iff \\
  (k + 6)y &= 4y &\iff \\
  k + 6 &= 4 \lor y = 0 \\
  &\texit{assume}\; y \neq 0 \\
  k &= -2 \;.
\end{align*}

Verifying:

\begin{align*}
  -2x_1 + x_2 &=  0  \; \land \\
  x_2         &= 4y  \; \land \\
  -x_1        &= -2y \; \land \\
  -x_1 - x_2  &= 6y \\
  \textit{is consitent} \iff \\
  x_1                   &= \frac{1}{2}x_2 \; \land \\
  x_2                   &= 4y             \; \land \\
  -\frac{1}{2}x_2       &= -2y            \; \land \\
  -\frac{1}{2}x_2 - x_2 &= -\frac{3}{2}x_2 = 6y \;.
\end{align*}

Hence whenever \(k = -2\), \(y(v_1+v_2)=v_3\).

\subsubsection{Answer 10}
\label{sec:orgheadline13}
Since \(u_i\) are the basis, none of them is a linear combination of the
others.  Hence \(v_1\) and \(v_2\) must ``compensate'' for the loss of \(u_4\).
In other words, whenever \(u_4\) is a linear combination of \(v_1\) and \(v_2\),
the set \(\{\vec{u_1}, \vec{u_2}, \vec{u_3}, \vec{v_1}, \vec{v_2}\}\) spans
\(\mathbb{R}^4\).

More formally, whenever:

\begin{align*}
  a_1(k\vec{u_1} - \vec{u_3} + \vec{u_4}) +
  a_2(\vec{u_1} + \vec{u_2} - \vec{u_4}) &= \vec{u_4} \iff \\
  (ka_1 + a_2)\vec{u_1} + a_2\vec{u_2} - a_1\vec{u_3} - (a_1 - a_2 - 1)\vec{u_4} &= 0 \iff \\
  \begin{bmatrix}
    k  & 1  & 0 \\
    0  & 1  & 0 \\
    -1 & 0  & 0 \\
    1  & -1 & 1
  \end{bmatrix} \textit{has solution}
\end{align*}


The above set spans \(\mathbb{R}^4\).

As before, we need to solve for \(k = 0\) and when it doesn't.

\begin{align*}
  \begin{bmatrix}
    0  & 1  & 0 \\
    0  & 1  & 0 \\
    -1 & 0  & 0 \\
    1  & -1 & 1
  \end{bmatrix}
  \begin{aligned} \xrightarrow{R_1 \; \textit{is redundatn}} \end{aligned}
  \begin{bmatrix}
    0  & 1  & 0 \\
    -1 & 0  & 0 \\
    1  & -1 & 1
  \end{bmatrix}
  \begin{aligned} \xrightarrow{R_1 = R_2, R_2 = R_1} \end{aligned}
  \begin{bmatrix}
    -1 & 0  & 0 \\
    0  & 1  & 0 \\
    1  & -1 & 1
  \end{bmatrix}
  \begin{aligned} \xrightarrow{R_3 = R_3 + R_1} \end{aligned} \\
  \begin{bmatrix}
    -1 & 0  & 0 \\
    0  & 1  & 0 \\
    0  & -1 & 1
  \end{bmatrix}
  \begin{aligned} \xrightarrow{R_3 = R_3 + R_2} \end{aligned}
  \begin{bmatrix}
    -1 & 0  & 0 \\
    0  & 1  & 0 \\
    0  & 0 & 1
  \end{bmatrix}
\end{align*}

In conclusion, whenever \(k = 0\), we can represent \(\vec{u_4}\) as a linear
combination of \(\vec{v_1}\) and \(\vec{v_2}\).

\(k \neq 0\) case:

\begin{align*}
  \begin{bmatrix}
    k  & 1  & 0 \\
    0  & 1  & 0 \\
    -1 & 0  & 0 \\
    1  & -1 & 1
  \end{bmatrix}
  \begin{aligned} \xrightarrow{R_4 = R_4 + R_3} \end{aligned}
  \begin{bmatrix}
    k  & 1  & 0 \\
    0  & 1  & 0 \\
    -1 & 0  & 0 \\
    0  & -1 & 1
  \end{bmatrix}
  \begin{aligned} \xrightarrow{R_3 = kR_3 + R_1} \end{aligned}
  \begin{bmatrix}
    k & 1  & 0 \\
    0 & 1  & 0 \\
    0 & 1  & 0 \\
    0 & -1 & 1
  \end{bmatrix}
  \begin{aligned} \xrightarrow{R_2 \; \textit{is redundant}} \end{aligned} \\
  \begin{bmatrix}
    k & 1  & 0 \\
    0 & 1  & 0 \\
    0 & -1 & 1
  \end{bmatrix}
  \begin{aligned} \xrightarrow{R_3 = R_3 + R_2} \end{aligned}
  \begin{bmatrix}
    k & 1 & 0 \\
    0 & 1 & 0 \\
    0 & 0 & 1
  \end{bmatrix}
\end{align*}

Thus, independent of \(k\), we will always be able to represent \(\vec{u_4}\) as
a linear combination of vectors \(\vec{v_1}\) and \(\vec{v_2}\).  Another way to
see this is to notice that both \(\vec{v_1}\) and \(\vec{v_2}\) have a component
from \(\vec{u_4}\) and this component cannot be cancelled by any other vector,
otherwise those other vectors wouldn't have formed a basis of \(\mathbb{R}^n\).

\subsection{Problem 5}
\label{sec:orgheadline18}
Let \(\vec{v}\), \(\vec{u_1},\dots,\vec{u_k}\) be vectors in \(\mathbb{R}^n\).
\(\vec{v}\) has a unique representation as a linear combination of vectors
\(\vec{u_1},\dots,\vec{u_k}\).

For questions (2) and (3) assume that for some \(w \in \mathbb{R}\), \(w =
   x_1\vec{u_1}+\dots+x_k\vec{u_k}\) has no solutions.

\begin{enumerate}
\item Prove \(\vec{u_1},\dots,\vec{u_k}\) are linearly independent.
\item Prove \(k < n\).
\item Prove \(\{w, \vec{u_1},\dots,\vec{u_k}\}\) is linearly independant.
\end{enumerate}

\subsubsection{Answer 11}
\label{sec:orgheadline15}
Assume, for contradiction, \(\vec{u_1},\dots,\vec{u_k}\) are linearly
dependent.  Then, there exist some \(\vec{u_n}\) s.t. for some
\(x_1\vec{u_1}+\dots+x_k\vec{u_k}=\vec{u_n}\).  Then, since \(\vec{u_n}\) is
used in the representation of \(\vec{v}\), we can write this represenation
in two distinct ways: one that involves \(\vec{u_n}\) and the other one
which doesn't.  However, we are given the representation is unique.

Hence, by contradiction, \(\vec{u_1},\dots,\vec{u_k}\) are linearly
independant.

\subsubsection{Answer 12}
\label{sec:orgheadline16}
Observe that \(k\) is at most \(n\), otherwise \(\vec{u_1},\dots,\vec{u_k}\) would
be linearly dependent.  (We proved this in \ref{sec:orgheadline15}.)

Assume, for contradiction \(k = n\), then \(\vec{u_1},\dots,\vec{u_k}\) spans
\(\mathbb{R}^n\), hence, every vector in \(\mathbb{R}^n\) is representable as a
linear combination of \(\vec{u_1},\dots,\vec{u_k}\).  However, we are given
that \(w\) is not representable as a linear combination of these vectors.

Hence, by contradition, \(k < n\).

\subsubsection{Answer 13}
\label{sec:orgheadline17}
The proof is immediate from the definition. \(w \neq
    x_1\vec{u_1}+\dots+x_k\vec{u_k}\), hence \(w\) is not a linear combination of
\(\vec{u_1},\dots,\vec{u_k}\), hence \(\{w, \vec{u_1},\dots,\vec{u_k}\}\) are
linearly independent.
\end{document}