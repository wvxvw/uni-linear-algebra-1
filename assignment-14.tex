% Created 2015-01-11 Sun 22:39
\documentclass[fleqn]{article}
\usepackage[utf8]{inputenc}
\usepackage[T1]{fontenc}
\usepackage{fixltx2e}
\usepackage{graphicx}
\usepackage{longtable}
\usepackage{float}
\usepackage{wrapfig}
\usepackage{rotating}
\usepackage[normalem]{ulem}
\usepackage{amsmath}
\usepackage{textcomp}
\usepackage{marvosym}
\usepackage{wasysym}
\usepackage{amssymb}
\usepackage{hyperref}
\tolerance=1000
\usepackage[utf8]{inputenc}
\usepackage[usenames,dvipsnames]{color}
\usepackage{a4wide}
\usepackage[backend=bibtex, style=numeric]{biblatex}
\usepackage{commath}
\usepackage{tikz}
\usepackage{amsmath}
\usetikzlibrary{shapes,backgrounds}
\usepackage{marginnote}
\usepackage{enumerate}
\usepackage{listings}
\usepackage{color}
\hypersetup{urlcolor=blue}
\hypersetup{colorlinks,urlcolor=blue}
\addbibresource{bibliography.bib}
\setlength{\parskip}{16pt plus 2pt minus 2pt}
\definecolor{codebg}{rgb}{0.96,0.99,0.8}
\DeclareMathOperator{\Sp}{Sp}
\DeclareMathOperator{\Solutions}{P}
\DeclareMathOperator{\Dim}{Dim}
\author{Oleg Sivokon}
\date{\textit{<2015-01-10 Sat>}}
\title{Assignment 14, Linear Algebra 1}
\hypersetup{
  pdfkeywords={Assignment, Linear Algebra},
  pdfsubject={Fourth asssignment in the course Linear Algebra 1},
  pdfcreator={Emacs 25.0.50.1 (Org mode 8.2.2)}}
\begin{document}

\maketitle
\tableofcontents


\lstset{ %
  backgroundcolor=\color{codebg},
  basicstyle=\ttfamily\scriptsize,
  breakatwhitespace=false,         % sets if automatic breaks should only happen at whitespace
  breaklines=false,
  captionpos=b,                    % sets the caption-position to bottom
  commentstyle=\color{mygreen},    % comment style
  framexleftmargin=10pt,
  xleftmargin=10pt,
  framerule=0pt,
  frame=tb,                        % adds a frame around the code
  keepspaces=true,                 % keeps spaces in text, useful for keeping indentation of code (possibly needs columns=flexible)
  keywordstyle=\color{blue},       % keyword style
  showspaces=false,                % show spaces everywhere adding particular underscores; it overrides 'showstringspaces'
  showstringspaces=false,          % underline spaces within strings only
  showtabs=false,                  % show tabs within strings adding particular underscores
  stringstyle=\color{codestr},     % string literal style
  tabsize=2,                       % sets default tabsize to 2 spaces
}

 \clearpage

\section{Problems}
\label{sec-1}

\subsection{Problem 1}
\label{sec-1-1}

Let $W$ and $U$ be subspaces of $\mathbb{R}^4[x]$:
\begin{align*}
  U &= \Sp \{ x^3 + 4x^2 - x + 3, x^3 + 5x^2 + 5, 3x^3 + 10x^2 + 5 \} \\
  W &= \Sp \{ x^3 + 4x^2 + 6, x^3 + 2x^2 - x + 5, 2x^3 + 2x^2 - 3x + 9 \}
\end{align*}


\begin{enumerate}
\item Find dimension, basis of $U$, $W$ and $U+W$.
\item What is the dimension and basis of $U \cap W$?
\item Find a subspace $T$ such that $T \oplus W = \mathbb{R}^4[x]$.
\end{enumerate}

\subsubsection{Answer 1}
\label{sec-1-1-1}
In order to find dimension, I can first find the basis, and then simply
count the number of vectors in it.  Since basis by definition has to be
linearly independent, I will adjoin vectors of $U$ to a matrix, will do
the same for vectors of $W$, triangularize these matrices and remove the
zero rows to obtain the basis.

\begin{multline*}
  \begin{bmatrix}
    1 & 2  & -1 & 3 \\
    1 & 5  & 0  & 5 \\
    3 & 10 & 0  & 5 \\
  \end{bmatrix}
  \xrightarrow{R_1 = R_2, R_2 = R_1}
  \begin{bmatrix}
    1 & 5  & 0  & 5 \\
    1 & 2  & -1 & 3 \\
    3 & 10 & 0  & 5 \\
  \end{bmatrix}
  \xrightarrow{R_3 = R_3 - 3R_1}
  \begin{bmatrix}
    1 & 5  & 0  & 5 \\
    1 & 2  & -1 & 3 \\
    0 & -5 & 0  & -10 \\
  \end{bmatrix} \\ \\
  \xrightarrow{R_2 = R_3, R_3 = R_2}
  \begin{bmatrix}
    1 & 5  & 0  & 5 \\
    0 & -5 & 0  & -10 \\
    1 & 2  & -1 & 3 \\
  \end{bmatrix}
  \xrightarrow{R_3 = R_3 - R_1}
  \begin{bmatrix}
    1 & 5  & 0  & 5 \\
    0 & -5 & 0  & -10 \\
    0 & -7  & -1 & -2 \\
  \end{bmatrix}
  \xrightarrow{R_3 = R_3 - R_2} \\ \\
  \begin{bmatrix}
    1 & 5  & 0  & 5 \\
    0 & -5 & 0  & -10 \\
    0 & -2 & -1 & 8 \\
  \end{bmatrix}
  \xrightarrow{R_3 = 5R_3}
  \begin{bmatrix}
    1 & 5   & 0  & 5 \\
    0 & -5  & 0  & -10 \\
    0 & -10 & -5 & 40 \\
  \end{bmatrix}
  \xrightarrow{R_3 = R_3 - 2R_2}
  \begin{bmatrix}
    1 & 5  & 0  & 5 \\
    0 & -5 & 0  & -10 \\
    0 & 0  & -5 & 20 \\
  \end{bmatrix}
\end{multline*}


Hence the basis of $U$ is $\{x^3+5x^2+5, -5x^2-10, -5x+20\}$. And its dimension is 3.

Similarly, for $W$:

\begin{multline*}
  \begin{bmatrix}
    1 & 4 & 0  & 6 \\
    1 & 2 & -1 & 5 \\
    2 & 2 & -3 & 9 \\
  \end{bmatrix}
  \xrightarrow{R_1 = R_2, R_2 = R_1}
  \begin{bmatrix}
    1 & 4  &  0 & 6 \\
    0 & -2 & -1 & -1 \\
    0 & 6  & -3 & -3 \\
  \end{bmatrix}
  \xrightarrow{R_3 = R_3 - 3R_1}
  \begin{bmatrix}
    1 & 4  &  0 & 6 \\
    0 & -2 & -1 & -1 \\
    0 & 0  & 0 & 0 \\
  \end{bmatrix}
\end{multline*}




Hence the basis of $W$ is $\{x^3+4x^2+6, -2x^2-x-1\}$, and its dimension is 2.

Now, let's find the basis of $V+U$.  Again, adjoin the bases of $U$ and $W$,
triangularize and remove all zero vectors if any.


\begin{multline*}
  \begin{bmatrix}
    1 &  5 &  0 &  5 \\
    0 & -5 &  0 & -10 \\
    0 &  0 & -5 &  20 \\
    1 &  4 &  0 &  6 \\
    0 & -2 & -1 & -1 \\
  \end{bmatrix}
  \xrightarrow{R_1 = R_2, R_2 = R_1}
  \begin{bmatrix}
    1 &  5 &  0 &  5 \\
    0 & -5 &  0 & -10 \\
    0 &  0 & -5 &  20 \\
    0 & -1 &  0 & -1 \\
    0 & -2 & -1 & -1 \\
  \end{bmatrix}
  \xrightarrow{R_3 = R_3 - 3R_1}
  \begin{bmatrix}
    1 &  5 &  0 &  5 \\
    0 & -5 &  0 & -10 \\
    0 &  0 & -5 &  20 \\
    0 & -1 &  0 & -1 \\
    0 &  0 & -1 &  1 \\
  \end{bmatrix} \\ \\
  \xrightarrow{R_3 = R_3 - 3R_1}
  \begin{bmatrix}
    1 &  5 &  0 &  5 \\
    0 & -1 &  0 & -1 \\
    0 &  0 & -5 &  20 \\
    0 & -5 &  0 & -10 \\
    0 &  0 & -1 &  1 \\
  \end{bmatrix}
  \xrightarrow{R_3 = R_3 - 3R_1}
  \begin{bmatrix}
    1 &  5 &  0 &  5 \\
    0 & -1 &  0 & -1 \\
    0 &  0 & -5 &  20 \\
    0 &  0 &  0 & -5 \\
    0 &  0 & -1 &  1 \\
  \end{bmatrix}
  \xrightarrow{R_3 = R_3 - 3R_1} \\ \\
  \begin{bmatrix}
    1 &  5 &  0 &  5 \\
    0 & -1 &  0 & -1 \\
    0 &  0 & -1 &  1 \\
    0 &  0 &  0 & -5 \\
    0 &  0 & -5 &  20 \\
  \end{bmatrix}
  \xrightarrow{R_3 = R_3 - 3R_1}
  \begin{bmatrix}
    1 &  5 &  0 &  5 \\
    0 & -1 &  0 & -1 \\
    0 &  0 & -1 &  1 \\
    0 &  0 &  0 & -5 \\
    0 &  0 &  0 &  15 \\
  \end{bmatrix}
  \xrightarrow{R_3 = R_3 - 3R_1}
  \begin{bmatrix}
    1 &  5 &  0 &  5 \\
    0 & -1 &  0 & -1 \\
    0 &  0 & -1 &  1 \\
    0 &  0 &  0 & -5 \\
    0 &  0 &  0 &  0 \\
  \end{bmatrix}
\end{multline*}

Since $\Dim(V)=4$ and $\Dim(U)+\Dim(W)=5$ it follows that $\Dim(U\cap W)$ has
to be 1.  This follows from the formula of sum of dimensions of subspaces, which
says $\Dim(V)=\Dim(S)+\Dim(T)-\Dim(S\cap T)$.
\subsection{Problem 2}
\label{sec-1-2}

Let $U$ and $W$ be subspaces of $\mathbb{R}^4$ such that $\Dim(U)>\Dim(W)$.
Provided that $W \cap U = \Sp \{(1, 2, 3, 4), (1, 1, 1, 1), (-1, 0, 1, 2)\}$,
and $(0, 0, 1, 0) \not \in U + W$.  Find the dimension of $U+W$ and its basis.
\subsection{Problem 3}
\label{sec-1-3}

Given the following subspaces of $\mathbb{R}^4$:
\begin{align*}
  U &= \Sp \{ (a, a-1, a, 4), (2, 2, 1, -3) \} \\
  W &= \Solutions\Big(\left.
    \begin{alignedat}{5}
      & x + & y +  & z  &{}={}& 0 \\
      & y + & 2z - & 2t &{}={}& 0 \\
    \end{alignedat}
  \right\}\Big)
\end{align*}

Where $\Solutions(X)$ is the vector space of all solutions of linear system $X$.

Find values of $a$ for which holds that $\Dim(U \cap W)=1$. Show the basis of
$U \cap W$ in this case.
\subsection{Problem 4}
\label{sec-1-4}

Prove or disprove each of the following statements:

\begin{enumerate}
\item If $B=\{\vec{v_1}, \ldots, \vec{v_n}\}$ is the basis of $V$ and $U \subseteq V$ is
a subspace of dimension $k$, $k \leq n$, then there are $k$ vectors in $B$ spanning
$U$.
\item If $V$ is a vectors pace of dimension $n$ and if $m \leq n, m \in \mathbb{N}$, then
exists sub-space $U$ of $V$ with dimension equal to $m$.
\end{enumerate}
\subsection{Problem 5}
\label{sec-1-5}

In field $\mathbb{F}$ are given members $a_1, a_2, \ldots, a_m$, not all zero and,
similarly, $b_1, b_2, \ldots, b_n$ not all zero.  What is the dimension of the
matrix given by:
\begin{align*}
  M     &= (m_{ij})_{1<i<m, 1<j<n} \\
  m_{ij} &= a_ib_j
\end{align*}
\subsection{Problem 6}
\label{sec-1-6}

Let $V$ be a vector space over $\mathbb{R}$ of dimension 3, and let $B$ be its basis.
Given vectors $\vec{v_1}, \vec{v_2}, \vec{v_3}$ and $\vec{w}$ in $V$:
\begin{equation*}
  [v_1]_B= \begin{pmatrix}  2 \\  3 \\  5 \\ \end{pmatrix},
  [v_2]_B= \begin{pmatrix}  1 \\ -2 \\ -3 \\ \end{pmatrix},
  [v_3]_B= \begin{pmatrix} -3 \\  2 \\ -1 \\ \end{pmatrix},
  [w]_B=   \begin{pmatrix}  5 \\  5 \\ 16 \\ \end{pmatrix}
\end{equation*}

Prove that $C=(\vec{v_1}, \vec{v_2}, \vec{v_3})$ is the basis of $V$ and find $[w]_C$.
% Emacs 25.0.50.1 (Org mode 8.2.2)
\end{document}